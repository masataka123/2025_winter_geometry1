\documentclass[dvipdfmx,a4paper,11pt]{article}
\usepackage[utf8]{inputenc}
%\usepackage[dvipdfmx]{hyperref} %リンクを有効にする
\usepackage{url} %同上
\usepackage{amsmath,amssymb} %もちろん
\usepackage{amsfonts,amsthm,mathtools} %もちろん
\usepackage{braket,physics} %あると便利なやつ
\usepackage{bm} %ラプラシアンで使った
\usepackage[top=20truemm,bottom=25truemm,left=22truemm,right=22truemm]{geometry} %余白設定
\usepackage{latexsym} %ごくたまに必要になる
\renewcommand{\kanjifamilydefault}{\gtdefault}
\usepackage{otf} %宗教上の理由でmin10が嫌いなので
%\usepackage{showkeys}\renewcommand*{\showkeyslabelformat}[1]{\fbox{\parbox{2cm}{ \normalfont\tiny\sffamily#1\vspace{6mm}}}}
\usepackage[driverfallback=dvipdfm]{hyperref}


\usepackage[all]{xy}
\usepackage{amsthm,amsmath,amssymb,comment}
\usepackage{amsmath}    % \UTF{00E6}\UTF{0095}°\UTF{00E5}\UTF{00AD}\UTF{00A6}\UTF{00E7}\UTF{0094}¨
\usepackage{amssymb}  
\usepackage{color}
\usepackage{amscd}
\usepackage{amsthm}  
\usepackage{wrapfig}
\usepackage{comment}	
\usepackage{graphicx}
\usepackage{setspace}
\usepackage{pxrubrica}
\usepackage{enumitem}
\usepackage{mathrsfs} 


\setstretch{1.2}


\newcommand{\R}{\mathbb{R}}
\newcommand{\Z}{\mathbb{Z}}
\newcommand{\Q}{\mathbb{Q}} 
\newcommand{\N}{\mathbb{N}}
\newcommand{\C}{\mathbb{C}} 
\newcommand{\Sin}{\text{Sin}^{-1}} 
\newcommand{\Cos}{\text{Cos}^{-1}} 
\newcommand{\Tan}{\text{Tan}^{-1}} 
\newcommand{\invsin}{\text{Sin}^{-1}} 
\newcommand{\invcos}{\text{Cos}^{-1}} 
\newcommand{\invtan}{\text{Tan}^{-1}} 
\newcommand{\Area}{\text{Area}}
\newcommand{\vol}{\text{Vol}}
\newcommand{\maru}[1]{\raise0.2ex\hbox{\textcircled{\tiny{#1}}}}
\newcommand{\sgn}{{\rm sgn}}
%\newcommand{\rank}{{\rm rank}}
\newcommand{\id}{{\rm id}}
\newcommand{\Sym}{{\rm Sym}}
\newcommand{\Supp}{{\rm Supp}}
\newcommand{\Ker}{{\rm Ker}}
\newcommand{\ima}{{\rm Im}}


\allowdisplaybreaks[4]
\usepackage{tcolorbox}
\tcbuselibrary{breakable, skins, theorems}

\theoremstyle{definition}
\newtheorem{thm}{定理}
\newtheorem{lem}[thm]{補題}
\newtheorem{prop}[thm]{命題}
\newtheorem{cor}[thm]{系}
\newtheorem{claim}[thm]{主張}
\newtheorem{dfn}[thm]{定義}
\newtheorem{rema}[thm]{注意}
\newtheorem{exa}[thm]{例}
\newtheorem{conj}[thm]{予想}
\newtheorem{prob}[thm]{問題}
\newtheorem{rem}[thm]{補足}

\DeclareMathOperator{\Ric}{Ric}
\DeclareMathOperator{\Vol}{Vol}
 \newcommand{\pdrv}[2]{\frac{\partial #1}{\partial #2}}
 \newcommand{\drv}[2]{\frac{d #1}{d#2}}
  \newcommand{\ppdrv}[3]{\frac{\partial #1}{\partial #2 \partial #3}}


\title{2024年度秋冬学期 大阪大学 理学部数学科\\  幾何学1演義 演習問題}
\author{岩井雅崇 (大阪大学)}
\date{2024年10月4日 \, ver 1.01}
%ここから本文.
\begin{document}

\maketitle
\tableofcontents
\newpage

\begin{center}
\setcounter{section}{-1}
\section{ガイダンス}
\label{sec-guide}
\end{center}

\begin{center}
{\Large 2024年度秋冬学期 大阪大学 理学部数学科\\  幾何学1 演義} \\
金曜4限(15:10-16:40) 理学部E310
\end{center}
\begin{flushright}
 岩井雅崇(いわいまさたか) \\
\end{flushright}
{\Large \underline{基本的事項}}
\begin{itemize}
  \setlength{\parskip}{0cm} % 段落間
  \setlength{\itemsep}{0cm} % 項目間
\item この授業は\underline{対面授業}です. \underline{金曜4限(15:10-16:40)に理学部E310}にて演習の授業を行います.
\item \underline{基本的には講義の授業とセットで受講してください.} 演義の授業のみ受講する場合は10月4日の授業後に申し出ること. 
\item 授業ホームページ(\url{https://masataka123.github.io/2024_winter_geometry1/})にて授業の問題等をアップロードしていきます. 
QRコードは下にあります.
\begin{figure}[htbp]
\begin{center}
 \includegraphics[height=30mm, width=30mm]{stokes.png}
 %\caption{授業のQRコード}
\end{center}
\end{figure}
\end{itemize}

\hspace{-18pt}{\Large \underline{成績に関して}}

次の1と2を満たしているものに単位を与えます.
\begin{enumerate}
  \setlength{\parskip}{0cm} % 段落間
  \setlength{\itemsep}{0cm} % 項目間
\item 幾何学1の講義の単位が可以上である.
\item 最終授業終了時までに0.1点以上の演習点(後述)を獲得していること.
\end{enumerate}
成績は演習点でつける予定ですが, 場合によっては講義の成績も参考にします.\footnote{理由としては成績がばらけるかどうかが, 現時点では予測が不可能だからです. }
%演習の成績は”講義の成績”+”演習点”×(点数補正係数)でつける予定です. 点数補正係数は実数かつ全員の成績から定まる係数です.

\medskip
\hspace{-18pt}{\Large \underline{演習点に関して}}

演習点を稼ぐには次の方法があります.
\begin{enumerate}
  \setlength{\parskip}{0cm} 
  \setlength{\itemsep}{0cm} 
\item レポートを提出する.  %レポートの出来により$0.1\sim0.5$点の演習点が与えられる.
\item 配布した演習問題を解き, その解答を黒板を用いて発表する. その場合の演習点は「解いた問題の難易度」と「発表の仕方・解答の方法」などから定まります.
\end{enumerate}

なお2の方が演習点は高めに設定しております.


\medskip
\hspace{-18pt}{\large \underline{1. レポートに関して}}

おそらく中間レポートと期末レポートを出します. レポート問題は演習問題の$^{\bullet}$がついてる問題(後述)の内容から出す予定です.
%(中間レポートは10-11月に, 期末レポートは12-1月に詳細を言う予定です.)

\newpage
%\medskip
\hspace{-18pt}{\large \underline{2. 黒板を用いた発表に関して}}

\hspace{-18pt}発表のルールは次のとおりです.
\begin{itemize}
  \setlength{\parskip}{0cm} 
  \setlength{\itemsep}{0cm} 
\item 問題の解答を黒板に書いて発表してください. 正答だった場合その問題はそれ以降解答できなくなります. %不正解だった場合他の人に解答権が移ります. 
\item  授業が始まる前にある程度演習問題をあらかじめ解き発表できる状態にしておいてください.
\item 複数人が解答したい問題があるときは平和的な手段で解答者を決めます.(例えば問題解答数が少ない人を優先する, トランプで決めるなどです. )
\item 発表方法があまりにも悪い場合(教科書丸写しなど)は減点します.
\end{itemize}

\hspace{-18pt}演習問題に関する注意点
\begin{itemize}
  \setlength{\parskip}{0cm} 
  \setlength{\itemsep}{0cm} 
\item \underline{演習問題は適当に出しているので, 全部解く必要はないです. } 解けない問題も多くあります.
\item 演習問題の難易度は一定ではないです. 問題番号の上に$^\bullet$や$^*$などの記号が書いてありますがこれは次を意味します.
\begin{enumerate}
  \setlength{\parskip}{0cm} 
  \setlength{\itemsep}{0cm} 
\item $^\bullet$がついてる問題は解けないといけない問題です. %演習点は$0.1\sim1.5$点です. ある程度授業を理解している人は他の人に解答を譲ってください.
\item 何もついてない問題は普通の問題です. ちょっと考えれば解ける範疇に収まっている(はずです). %演習点は$0.5\sim2.5$点です.
\item $^*$問題や$^{**}$問題は難しそうな問題です. ちょっと難しい問題から激ムズの問題まであります. 私やTAが解けない問題もあります. 基本的に解かせる気はなく自由気ままに出しております. %演習点は$1.5 \sim 7$点です. 
\end{enumerate}
難易度が高い問題を解いた場合や解答が素晴らしい場合は演習点を多くもらえます.
\end{itemize}

\hspace{-18pt}次のご協力をお願いいたします.
\begin{itemize}
  \setlength{\parskip}{0cm} 
  \setlength{\itemsep}{0cm} 
\item 発表後, スマホ等で黒板にある解答を撮影しても構いません. (ただし黒板のみを撮影してください) %理由としては私の方で解答を用意してないからです. 
解答者も撮影のご協力お願いします.
\item 板書は他人が読めるように, 文字の大きさ・綺麗さ・板書の量に配慮してください. 字は汚くてもいいので, 最低限読めるようにしてください. %(私は文字を綺麗に板書できないので, 相当汚い字でも読むことはできますが…)
\end{itemize}


\medskip
\hspace{-18pt}{\Large \underline{まとめ}}
\begin{enumerate}
  \setlength{\parskip}{0cm} 
  \setlength{\itemsep}{0cm} 
\item \underline{単位が欲しい方はレポートを提出し, 講義で可以上を取ってください.} %単位だけ欲しい人は一回も黒板で発表せずにレポートを2回提出してください. さらに位相空間論の講義の試験で可以上をもらってください. それで演義の成績の単位ももらえます. (講義の試験が良ければ演義の成績も良いです.)
\item ちょっと欲張りな人は$^\bullet$がついている問題や何もついてない問題を発表してください. なお$^{\bullet}$がついている問題が全て解ければ, 講義の試験の単位は(おそらく)もらえると思います.
\item 意欲のある人は難しい問題など色々解いてください. そのほうが私は楽しいです.
\end{enumerate}




\vspace{11pt}
\hspace{-18pt}{\Large \underline{休講予定・その他}}
\begin{itemize}
  \setlength{\parskip}{0cm} % 段落間
  \setlength{\itemsep}{0cm} % 項目間
  \item 休講予定: 2024年12月13日.(大阪大学で開催する研究集会の世話人のため)  \footnote{他にあるとすれば2024年11月15日です. また授業回数が少ない場合は補講をするかもしれません.} 
  %\item 休講予定: 2024年1月16日 (休講はほぼ確定) 2023年12月05日または2023年12月12日 (どちらか休講にするかも・未確定)
    %\item 演習問題と授業内容が噛み合ってない可能性があります.
    \item 問題のミスがあれば私に言ってください. ミスはかなりあると思います. 
  \item 休講情報や演習問題の修正をするので, こまめにホームページを確認してください.
    \item オフィスアワーを月曜16:00-17:00に設けています. この時間に私の研究室に来ても構いません(ただし来る場合は前もって連絡してくれると助かります.)
    \item TAさんは演義の時間中に巡回しているので, 自由にご質問して構いません. 
    %\item $\pi$-base \url{https://topology.jdabbs.com}も活用してください. 
 \end{itemize}
 
\newpage






\begin{center}
\section{多様体の復習}
\label{sec-manifold}
\end{center}
\begin{flushright}
 岩井雅崇(いわいまさたか)
\end{flushright}



\begin{tcolorbox}[
    colback = white,
    colframe = green!35!black,
    fonttitle = \bfseries,
    breakable = true]
    \begin{dfn}[]
    \label{defn_local}
    位相空間$M$の開集合$U$から$\R^m$の開集合$V$への同相写像$\varphi : U \rightarrow V$について$(U, \varphi)$を$m$次元座標近傍といい, $\varphi$を$U$上の局所座標系という. 
    
    $p \in U$について, $\varphi(p) =(x_1, \ldots, x_m)$とかける. $x_1, \ldots, x_m$を$(U, \varphi)$に関する$p$の局所座標という. ($(U, \varphi)$のことを$(U; x_1, \ldots, x_m)$と書くこともある.) 
    \end{dfn}
    
    \end{tcolorbox}
    \begin{tcolorbox}[
    colback = white,
    colframe = green!35!black,
    fonttitle = \bfseries,
    breakable = true]
    \begin{dfn}[]
    $M$をハウスドルフ空間とする. 次の条件が成り立つとき$M$は$m$次元$C^{\infty}$級多様体と呼ばれる.
     \begin{enumerate}
     \setlength{\parskip}{0cm}
  \setlength{\itemsep}{0pt} 
     \item 座標近傍系$\{(U_\lambda, \varphi_\lambda)\}_{\lambda \in\Lambda}$があって, $M = \cup_{\lambda \in \Lambda} U_{\lambda}$となる. 
     \item $U_\lambda \cap U_\mu \neq \phi$なる$\lambda, \mu \in \Lambda$について
    $
   \varphi_\mu\circ \varphi_{\lambda}^{-1} : \varphi_{\lambda}(U_\lambda \cap U_\mu) \rightarrow \varphi_{\mu}(U_\lambda \cap U_\mu) 
    $
    は$C^{\infty}$級写像である. 
     \end{enumerate}
    \end{dfn}
    \end{tcolorbox}    

\begin{tcolorbox}[
    colback = white,
    colframe = green!35!black,
    fonttitle = \bfseries,
    breakable = true]
    \begin{dfn}[接ベクトル空間]
 $(U; x_1, \ldots, x_m)$を$p$の周りの座標系とする. このとき$(\pdrv{}{x_i})_{p}$を$p$の開近傍で定義された$C^{\infty}$級関数$\xi$について実数
     $$
   \left(\pdrv{}{x_i}\right)_{p} :   \xi \mapsto \pdrv{\xi}{x_i}(p)
     $$
     を対応させるものとする. $m$個の$(\pdrv{}{x_1})_{p}, \ldots, (\pdrv{}{x_m})_{p}$ではられる$\R$ベクトル空間を$T_{p}M$と表し\underline{$M$の接ベクトル空間}と呼ぶ. 
    \end{dfn}
    \end{tcolorbox}    
    
    \begin{rem}
多様体の基礎によると$T_{p}M$の元を表す方法は他にもある. 今回は簡単な定義に基づいた.  つまり$v \in T_{p}M$の元はある$a_1, \ldots, a_m \in \R$を用いて
     $$
     v = \sum_{i=1}^{m} a_i \left(\pdrv{}{x_i}\right)_{p} $$
     と書くことができる.\footnote{接ベクトル空間を「何かよくわからないもの$(\pdrv{}{x_i})_{p}$が$\R$上ではられるもの」と思うという荒技もある. これはベクトル束の立場から見るとそうなる. 接ベクトル空間の厳密な定義は意外と難しい.}
          
    \end{rem}
    

\begin{tcolorbox}[
    colback = white,
    colframe = green!35!black,
    fonttitle = \bfseries,
    breakable = true]
    \begin{dfn}[]
    \label{differential}
    $M$を$m$次元$C^{\infty}$級多様体, $N$を$n$次元$C^{\infty}$級多様体, $f: M \rightarrow N$を$C^{\infty}$級写像とする. 
    $p \in M$をとり$q := f(p) \in N$とする. 
$(V, y_1, \ldots, y_n)$を$q$の周りの座標系, $(U; x_1, \ldots, x_m)$を$f(U) \subset V$となる$p$の周りの座標系とする.
         $f$を$(U; x_1, \ldots, x_m)$と$(V, y_1, \ldots, y_n)$ によって局所座標表示したものを
$$y_1=f_1(x_1, \ldots, x_m), \ldots, y_n=f_n(x_1, \ldots, x_m)$$
としたとき, $(df)_{p} : T_{p}(M) \rightarrow T_{q}(N)$を次のように定義する.
$$ (df)_{p} : \sum_{i=1}^{m} a_i \left(\pdrv{}{x_i}\right)_{p}  \mapsto 
\sum_{j=1}^{n} \left(\sum_{i=1}^{m} a_i  \pdrv{f_{j}}{x_i}(p) \right)  \left(\pdrv{}{y_j}\right)_{q} $$
     
この$(d f)_{p} : T_{p}(M) \rightarrow T_{q}(N)$を\underline{$p$における$f$の微分}という.
    \end{dfn}
    \end{tcolorbox} 
    
    
    \begin{rem}
 多様体の基礎によると, これ以外の定義もある.    
また定義\ref{differential}において$b_j = \sum_{i=1}^{m} a_i  \pdrv{f_{j}}{x_i}(p) $ とおき, $n \times m$行の行列$(Jf)_{p}$を
   $$
   (Jf)_{p} = 
   \begin{pmatrix}
   \pdrv{f_{1}}{x_1}(p) & \pdrv{f_{1}}{x_2}(p)  & \cdots & \pdrv{f_{1}}{x_m}(p) \\
   \vdots& \vdots& \cdots & \vdots \\
   \pdrv{f_{n}}{x_1}(p) & \pdrv{f_{n}}{x_2}(p)  & \cdots & \pdrv{f_{n}}{x_m}(p) \\
   \end{pmatrix}
   \text{とすれば,} 
   \begin{pmatrix}
   b_1 \\ \vdots \\ b_n 
   \end{pmatrix}
   =
   (Jf)_{p} 
   \begin{pmatrix}
   a_1 \\ \vdots \\ a_m
   \end{pmatrix}
      \text{が成り立つ.} 
   $$
 $(Jf)_{p} $をヤコビ行列と呼ぶ.\footnote{これは座標系$(U; x_1, \ldots, x_m),(V, y_1, \ldots, y_n)$に依存する. } 
%またここでも定義\ref{defn_local}のような同一視がなされている. 正確に書けば次のとおりである:
%座標系を$(U, \varphi)=(U; x_1, \ldots, x_m)$とする. $\varphi(U) \subset \R^{m}$の標準座標を$r_1, \ldots, r_m$とする. $(V, \psi) = (V, y_1, \ldots, y_n)$を$q$の座標系とする. $\psi(z) = (y_1(z), \ldots, y_n(z))$に注意すれば, $$  \pdrv{f_j}{x_i}(p) := \pdrv{ (y_j \circ f  \circ \varphi^{-1})}{r_i}( \varphi(p) )  \text{となる.}$$

    \end{rem}






\begin{tcolorbox}[
    colback = white,
    colframe = green!35!black,
    fonttitle = \bfseries,
    breakable = true]
    \begin{dfn}[埋め込みとはめ込み]
    $f : M \rightarrow N$を多様体の間の$C^{\infty}$級写像とする. 
    \begin{itemize}
         \setlength{\parskip}{0cm}
  \setlength{\itemsep}{0pt} 
    \item $f$が\underline{はめ込み}であるとは, 任意の点$p \in M$について微分写像$(df)_{p} : T_{p}(M) \rightarrow T_{f(p)}(N)$が単射であること.
    \item $f$が\underline{埋め込み}であるとは, $f$がはめ込みであり, $f : M \rightarrow f(M)$が同相であることとする. ここで$f(M)$には$N$の相対位相を入れる. このとき$f(M)$は$N$の部分多様体であることが知られている. 
    \end{itemize}

    \end{dfn}
    \end{tcolorbox}
    
 \begin{tcolorbox}[
    colback = white,
    colframe = green!35!black,
    fonttitle = \bfseries,
    breakable = true]
    \begin{thm}[][多様体の基礎 定理15-1]
    \label{thm-regular}
    
$f : M \rightarrow N$を多様体の間の$C^{\infty}$級写像とする. さらに$q \in N$を正則値であると仮定する. 
$f^{-1}(q) \neq \varnothing $ならば, $f^{-1}(q) $は$\dim M - \dim N$次元の$C^\infty$級部分多様体である.

ここで$q \in N$が$f : M \rightarrow N$の正則値であるとは, 任意の$p \in f^{-1}(q)$について,
微分写像
$$
(df)_{p} : T_{p}(M) \rightarrow T_{f(p)}(N)
$$
が全射であることとする.
    \end{thm}
    \end{tcolorbox}
    



%\hspace{-22pt}{$\bullet$ 多様体の例}
\begin{enumerate}[label=\textbf{問}\ref*{sec-manifold}.\arabic*]

\item $^\bullet$ \label{sphere} $S^{n} := \{ (x_1, x_2, \ldots, x_{n+1}) \in \R^{n+1}| \sum_{i=1}^{n+1} x_{i}^{2} = 1 \}$とおく. 
%$S^{m}$が$m$次元の$C^{\infty}$級多様体であることを2通りの方法で示したい. 次の問いに答えよ. 
% $N=(0,0,\ldots, 1),S=(0,0,\ldots, -1)$とし, $U_{N} = S_{m} \setminus N$, $U_{S} = S_{m} \setminus S$とおく. $$
%\begin{array}{ccccc}
%\varphi_{N}: &U_{N}& \rightarrow & \R^{m} & \\&(x_{1},x_{2}, \ldots ,x_{m+1})& \longmapsto &(\frac{x_{1}}{1-x_{m+1}}, \ldots, \frac{x_{m}}{1-x_{m+1}})&\end{array}$$	
%$$\begin{array}{ccccc}\varphi_{S}: &U_{S}& \rightarrow & \R^{n} & \\&(x_{1}, x_{2}, \ldots, x_{n+1})& \longmapsto &(\frac{x_1}{1 + x_{m+1}}, \ldots, \frac{x_m}{1+x_{m+1}})&\end{array}$$とおく. $\{(U_N, \varphi_N), (U_S, \varphi_S) \}$が$S^m$の座標近傍系を与えることを示し, これにより
$S^n$の座標近傍系を具体的に構成することにより, $S^{n}$は$n$次元の$C^{\infty}$級多様体となることを示せ. \footnote{$2n + 2$個の座標近傍系で作る方法と, $2$個の座標近傍系で作る方法がある. 前者の方が簡単である. なお座標近傍系$(U \varphi)$に関して$\varphi$が同相であることは示さなくても良い.}
	
\item $^\bullet$ $f : \R^{n+1} \rightarrow \R$となる$C^{\infty}$級写像で$f^{-1}(1) = S^{n}$かつ$1 \in \R$が$f$の正則値であるようなものを一つ求めよ. またこれを用いて$S^{n}$は$n$次元の$C^{\infty}$級多様体であることを示せ. 

\item  \label{real_plane} $^\bullet$ $\R^{n+1} \setminus \{ 0\}$について, 同値関係$\sim$を
	$$
	x \sim y \Leftrightarrow \text{0でない実数$\alpha$が存在して$x = \alpha y$}
	$$
	と定義する.
	$ \R\mathbb{P}^{n}:= \R^{n+1} \setminus \{ 0\}/\sim$と書き実射影空間と呼ぶ. %$ \R\mathbb{P}^{n}$は$n$次元$C^{\infty}$級多様体となることが知られている.  
	以下$x= (x_{1}, x_{2}, \ldots, x_{n+1})$を$\R\mathbb{P}^{n}$の元とみなしたものを$(x_{1}: \cdots : x_{n+1})$と書き実同次座標と呼ぶ. 

 $U_{i} = \{ (x_{1}:x_{2}: \ldots : x_{n+1}) | x_{i}\neq 0\}$とおき, 
$$
\begin{array}{ccccc}
\varphi_{i}: &U_{i}& \rightarrow & \R^{n} & \\
&(x_{1}:x_{2}: \ldots : x_{n+1})& \longmapsto &(\frac{x_1}{x_i}, \ldots, \frac{x_{i-1}}{x_i}, \frac{x_{i+1}}{x_i}, \ldots, \frac{x_n}{x_i})&
\end{array}
$$	
と定める. $\{ (U_i , \varphi_{i})\}_{i=1}^{n+1}$は座標近傍系となることを示し, $\R \mathbb{P}^{n}$は$n$次元の$C^{\infty}$級多様体であることを示せ. ただし$\R \mathbb{P}^{n}$がハウスドルフ空間および$\varphi_i$が同相写像であることは認めて良い. 


%\item 
%	$$\begin{array}{ccccc}\pi: &S^{n}& \rightarrow & \R\mathbb{P}^{n}& \\&(x_{1}, \ldots, x_{n+1}) & \longmapsto & (x_{1}: \cdots : x_{n+1})&\end{array}$$は全射$C^{\infty}$級写像であることをしめせ.
%\item 任意の$q \in \R\mathbb{P}^{n}$について$f^{-1}(q)$の個数を求めよ.
%\item $f : S^2 \rightarrow \R^3$を$f(x,y,z)=(yz,zx,xy)$とする. $f$と$\pi$を使って自然に$\tilde{f}: \R\mathbb{P}^{2} \rightarrow \R^3$が定義できることを示せ. 
%\item $\tilde{f}$ははめ込みではないことをしめせ.
%\end{enumerate}


\item $^\bullet$ $M(n,\R)$を$n\times n$行列の全体の集合とする.  $M(n,\R)$を$\R^{n^2}$と同一視する. 特に$M(n,\R)$が$n^2$次元$C^{\infty}$級多様体となる. 次の問いに答えよ.
	\begin{enumerate}
	     \setlength{\parskip}{0cm}
  \setlength{\itemsep}{0pt} 
	%\item $M(n,\R)$は$\R^{n^2}$と同一視できることを示せ. 特に$M(n,\R)$が$n^2$次元$C^{\infty}$級多様体となる.
	\item $GL(n, \R) = \{ A \in M(n,\R) | \det A \neq 0\}$が$C^{\infty}$級多様体であることを示し, その次元を求めよ. 
	\item $SL(n, \R) = \{ A \in M(n,\R) | \det A =1\}$が$C^{\infty}$級多様体であることを示し, その次元を求めよ. 
	\end{enumerate}


\item $^\bullet$ $f : \R \mathbb{P}^{n} \rightarrow \R$を
$$
f([x_1: \cdots : x_{n+1}] ) = \frac{x_{1}^{2}}{x_{1}^{2} + \cdots+ x_{n+1}^{2}}
$$
とおく. 次の問いに答えよ.
\begin{enumerate}
     \setlength{\parskip}{0cm}
  \setlength{\itemsep}{0pt} 
\item $f$がwell-definedな$C^{\infty}$級写像であることを示せ.
\item $(df)_{p}$が消える$p \in \R \mathbb{P}^{n}$の点を全て求めよ. (ヒント: \ref{real_plane}における座標近傍系$\{ (U_i , \varphi_{i})\}_{i=1}^{n+1}$について, $f \circ \varphi_{i}^{-1} : \R^n \to \R$のヤコビ行列を計算せよ. )
\item $f$の最大値・最小値を求めよ
\end{enumerate}

\item 
	$$
\begin{array}{ccccc}
f: &S^{3}& \rightarrow & \R & \\
&(x,y,z,w) & \longmapsto & xy - zw&
\end{array}
$$
とおく.  $f^{-1}(0)$は$S^{3}$の部分多様体であることをしめせ. (ヒント: \ref{sphere}の座標近傍系$\{ (U_i , \varphi_{i})\}$を用いて, $f \circ \varphi_{i}^{-1}$のヤコビ行列を計算する. )



\item  (多様体の基礎 11章) $\C^{2} \setminus \{ 0\}$について, 同値関係$\sim$を
	$$
	z \sim w \Leftrightarrow \text{0でない複素数$\alpha$が存在して$z = \alpha w$}
	$$
	と定義する.$ \C\mathbb{P}^{1}:= (\C^{2} \setminus \{ 0\})/\sim$と書き複素射影空間と呼ぶ. 以下$(z, w)$を$\C\mathbb{P}^{1}$の元とみなしたものを$(z :  w)$と書き複素同次座標と呼ぶ.
	次の問いに答えよ.
	\begin{enumerate}
	     \setlength{\parskip}{0cm}
  \setlength{\itemsep}{0pt} 
%\item $U_{= \{ (z_{1}:z_{2}: \ldots : z_{n+1}) | z_{i}\neq 0\}$とおき, $$\begin{array}{ccccc}\varphi_{i}: &U_{i}& \rightarrow & \C^{n} & \\&(z_{1}:z_{2}: \ldots : z_{n+1})& \longmapsto &(\frac{z_1}{z_i}, \ldots, \frac{z_{i-1}}{z_i}, \frac{z_{i+1}}{z_i}, \ldots, \frac{z_n}{z_i})&\end{array}$$	と定める. $\{ (U_i , \varphi_{i})\}_{i=1}^{n+1}$は座標近傍系となることを示し,
\item  $\C\mathbb{P}^{1}$が(実)$2$次元の$C^{\infty}$級多様体であることを示せ. ただし$\C \mathbb{P}^{1}$がハウスドルフ空間であることは認めて良い. 
\item $i :\C \rightarrow \C\mathbb{P}^{1}$を$i(z) = (z:1)$とすることにより, $\C$を$\C\mathbb{P}^{1}$の開部分多様体と見なす.  $f : \C \rightarrow \C$を$f(z) = z^2 +1$とおく. このときある$F : \C\mathbb{P}^{1} \rightarrow \C\mathbb{P}^1$となる$C^{\infty}$級写像で$F|_{\C} = f$となるものがあることを示せ. 
	\end{enumerate}
	
	

\item (多様体の基礎 15章) $k,n$を$1 \le k \le n$となる自然数とし$M_{k, n}$を実数係数$k \times n$行列全体とする.
$$
V_{k,n}= \{ A \in M_{k, n}| A ({}^{t}A) = E\}
$$
とする. 次の問いにこたえよ.
	\begin{enumerate}
	     \setlength{\parskip}{0cm}
  \setlength{\itemsep}{0pt} 
	\item $f : \R^{2n} \rightarrow \R^3$を次で定める.
$$
\begin{array}{ccccc}
f: &\R^{2n}& \rightarrow & \R^{3} & \\
&(x_{1}, \ldots, x_n, y_1, \ldots, y_n) & \longmapsto & 
(\sum_{i=1}^{n} x_{i}^{2}, \sum_{i=1}^{n} y_{i}^{2}, \sum_{i=1}^{n} x_{i}y_{i})&
\end{array}
$$
	$(x_{1}, \ldots, x_n, y_1, \ldots, y_n) \in \R^n$での$f$のヤコビ行列を求めよ
	\item $V_{2,n}$は$\R^{2n}$の$C^{\infty}$級部分多様体であることを示し, その次元を求めよ.
	\item$V_{3,n}$は$\R^{3n}$の$C^{\infty}$級部分多様体であることを示し, その次元を求めよ.
	\end{enumerate}

%\item$^{**}$ 実ベクトル空間$\R^{n}$について, その$k$次元ベクトル部分空間全体の集合を$G_{n,k}$とおく. $G_{n,k}$は自然に$C^{\infty}$級多様体の構造を持つことを示し, その次元を求めよ. この多様体はグラスマン多様体と呼ばれる. 

\item$^{**}$ $1 \leqq k < n$となる自然数について, 
$A_{k, n}$を$k \times n$実数行列でランクが$k$となる行列全体の集合とし, $\R^{kn}$の部分集合とみなすことで$A_{k,n}$に$\R^{kn}$の相対位相を入れる. 
$A_{k, n}$に同値関係$\sim$を
$$
	A \sim B \Leftrightarrow \text{正則な$k \times k$実数行列$G$が存在して$A = GB$}
$$
と定義する. $G_{k,n}:= A_{k, n}/\sim$と書き実グラスマン多様体と呼ぶ. $G_{n,k}$は$C^{\infty}$級多様体の構造を持つことを示し, その次元を求めよ. \footnote{難しければ$n=4, k=2$の場合を解答しても良い.}

%\end{enumerate}	
%\hspace{-22pt}{$\bullet$ 多様体間の$C^{\infty}$級写像}
%\begin{enumerate}[label=\textbf{問}\ref*{sec-manifold}.\arabic*]
%\setcounter{enumi}{9}

\item $^\bullet$ $M$を$m$次元コンパクト$C^{\infty}$級多様体とする. $C^{\infty}$級写像$f: M \rightarrow \R^{m}$ではめ込みとなるものは存在しないことを示せ. ($m = \dim M$に注意すること).

\item $^\bullet$ $M$と$N$が微分同相であるならば$\dim M =\dim N$を示せ. 

%\item $M$を$m$次元$C^{\infty}$級多様体とし, $f$を$M$上の$C^{\infty}$級関数とする. ある点$p \in M$において$(df)_{p} \in T^{*}M$は$(df)_{p} \neq 0$を満たすとする. このとき$C^{\infty}$級可微分曲線$\varphi : (-1,1) \rightarrow M$で$\varphi(0) = p$かつ$(f \circ \varphi)'(0) >0$なるものが存在する.  $M$がコンパクトならば$(df)_{p} =0$なる$p \in M$が存在する.

\item $^\bullet$ $f : M\rightarrow \R$を$C^{\infty}$級写像とする.
\begin{enumerate}
\item $p \in M$において$(df)_{p} \neq 0$ならば, ある$C^{\infty}$級写像$c : (-1,1) \rightarrow M$で$c(0)=p$かつ$(f \circ c)'(0) >0$となるものが存在することを示せ. 
\item $M$がコンパクトならば$(df)_{p} = 0$となる$p \in M$が存在することを示せ. 
\end{enumerate}


\item$^*$ 次の問いに答えよ
\begin{enumerate}
     \setlength{\parskip}{0cm}
  \setlength{\itemsep}{0pt} 
\item $f : \R^m \to \R^n$を$C^\infty$級写像とする. 任意の$p \in \R^m$について$f$のヤコビ行列$(Jf)_{p} $が零行列であるならば, $f$は定値写像であることを示せ. 
\item $M,N$を連結な$C^\infty$級多様体とし, $f : M \rightarrow N$を$C^\infty$級写像とする. 任意の$p \in M$について$(df)_{p} : T_{p}(M) \rightarrow T_{f(p)}(N)$が零写像であるならば, $f$は$M$を$N$の一点へ写す定値写像であることを示せ. 
\end{enumerate}

\item$^*$ %次の問いに答えよ. ただし$m, n$は$m \ge n$である1以上の整数とする. 
%\begin{enumerate}
  %   \setlength{\parskip}{0cm}
  %\setlength{\itemsep}{0pt} 
%\item $f : \R^m \to \R^n$を$C^\infty$級写像とする. 任意の$p \in \R^m$について$\rank (Jf)_{p}=n$が成り立つならば, $f(\R^m)$は開集合であることを示せ. 
%\item 
$M,N$をそれぞれ$m$次元, $n$次元の$C^{\infty}$多様体とし$C^{\infty}$写像$f : M \rightarrow N$とする. さらに$m \ge n$, $M$はコンパクト, $N$は連結であるとする. 
任意の$p \in M$について$(df)_{p} : T_{p}(M) \rightarrow T_{f(p)}(N)$が全射であるとき$f$も全射であることを示せ. 
%\end{enumerate}


	





%\item $m,k$を正の自然数とする. $C^{\infty}$級写像$f : \R^{m+k} \rightarrow \R^{k}$とその正則値$c$を考える. $M = f^{-1}(c)$は$\R^{m+k}$の部分多様体となり, 任意の$p \in M$について$T_{p}M = {\rm Ker}(df)_{p}$となることを示せ. またこれを用いて問題1.5を示せ. (つまり$a \in S^{m}$について$T_{a}S^{m} = \{ v \in \R^{m+1} | <a,v> = 0\}$となることを示せ. ここで$<\bullet, \bullet>$は$ \R^{m+1}$上のユークリッド内積とし, $T_{a}\R^{m+1}$と$\R^{m+1}$を同一視する.)





%次の問いに答えよ.	
%	\begin{enumerate}
%	\item $z_1, \ldot, z_{n}$を$\C^{n}$の座標とする. $1 \ge \alpha_1 < \cdots < \alpha_{k} \ge n$となる自然数の集合$\alpha = (\alpha_1,  \ldots, \alpha_{k})$について
%	$$U_{\alpha} :~ \{ V \in G_{n,k} | \text{$z_{\alpha_1}, \ldots, z_{\alpha_{k}}$は$V$上で一次独立}\}$$
%	とおく. 
	
%	\item $G_{n,k}$は自然に$C^{\infty}$級多様体の構造を持つことを示し, その次元を求めよ.
%\end{enumerate}

\end{enumerate}	

%%%%%%%%%%%%%%%%%%%%%%%%%%5
\begin{comment}


\hspace{-22pt}{$\bullet$ ベクトル場・積分曲線}

\begin{enumerate}[label=\textbf{問}\ref*{sec-manifold}.\arabic*]
\setcounter{enumi}{14}



\item $\R^{2}$上のベクトル場を$X = -y \pdrv{}{x} + x \pdrv{}{y}$とする. 次の問いに答えよ.
\begin{enumerate}
\item $X$は完備であることを示せ.%\footnote{極座標を用いたら幾分楽かもしれない.}
 %$(1,0)$を通る極大積分曲線を求めよ.
\item $\{ \varphi_{t} \}_{t \in \R}$を1パラメーター変換群とする. $ \varphi_{t}: M \rightarrow M$を求めよ. 
\item $X_{p} =  \drv{\varphi_{t}(p)}{t}\Bigr|_{t=0} $を確かめよ.
\end{enumerate}

\item $^{*}$ $M$をコンパクト$C^{\infty}$級多様体とし$X$を$C^{\infty}$級ベクトル場とする. 
$M$上の$C^{\infty}$級関数$f,g : M \rightarrow \R$が$Xf = g, Xg =f$を満たすとする.
次の問いに答えよ.
\begin{enumerate}
\item $X$の任意の積分曲線$c : \R \rightarrow M$について$(f \circ c)'' (t) = (f \circ c) (t)$であることを示せ.
\item $f,g$は恒等的に0であることを示せ. 
\end{enumerate}


\end{enumerate}
\end{comment}
%%%%%%%%%%%%%%%%%%%%%%%%%%%%%%
\newpage



\begin{center}
\section{$\R^n$上の微分形式}
\label{sec-Rn-diff}
\end{center}
\begin{flushright}
 岩井雅崇(いわいまさたか)
\end{flushright}

 \begin{tcolorbox}[
    colback = white,
    colframe = green!35!black,
    fonttitle = \bfseries,
    breakable = true]
\begin{dfn}
$\R^n$の開集合$U$上の$k$次微分形式とは, 
    $$
f d x_{i_1}\wedge dx_{i_2} \wedge \cdots \wedge dx_{i_k}
    $$
    の有限和としてかけるものとする. $f : U \rightarrow \R$は$C^{\infty}$級関数とする.     
   \end{dfn}
    \end{tcolorbox}
   \begin{rem}
   厳密には
   \begin{itemize}
    \setlength{\parskip}{0cm}
  \setlength{\itemsep}{2pt} 
   \item $p \in U$について$(dx_i)_p$は余接ベクトル空間$T_{p}^{*}U$(接ベクトル空間$T_{p}U$の双対空間)の元
   \item $dx_i$は$U$上の1次微分形式 
   \item  $\omega = \sum_{ 1 \le i_1< \cdots < i_k\le m }f_{i_1 i_2 \cdots i_k}d x_{i_1}\wedge dx_{i_2} \wedge \cdots \wedge dx_{i_k}$ は$U$上の$k$次微分形式
   \end{itemize}
   となる. 厳密な定義は授業や次回の演習にすることにして, 今回の演習では"厳密なことはあんまり考えず"微分形式の計算をできることを目標とする. 
   \end{rem}

以下$k$次微分形式は$\R^n$の開集合$U$上のものを考えるとする. 
 \begin{tcolorbox}[
    colback = white,
    colframe = green!35!black,
    fonttitle = \bfseries,
    breakable = true]
\begin{dfn}[微分形式の計算規則]
\text{}
   \begin{itemize}
    \setlength{\parskip}{0cm}
  \setlength{\itemsep}{2pt} 
\item $0 dx_1 =0$. $k$次微分形式でも同様. 
\item  $f dx_1 \pm g dx_1 = (f \pm g) dx_1$ $k$次微分形式でも同様. 
\item  $dx_i \wedge dx_j = - dx_j \wedge dx_i$. $k$次微分形式においても$dx_i$と$dx_j $の順番を入れ替えると, $-1$倍される.
\item $dx_i \wedge dx_i=0 $. $k$次微分形式$\omega$においても$dx_i \wedge dx_i$というものがあれば$\omega=0$となる. 
\end{itemize}
 \end{dfn}
    \end{tcolorbox}
    
    
 \begin{tcolorbox}[
    colback = white,
    colframe = green!35!black,
    fonttitle = \bfseries,
    breakable = true]
\begin{dfn}[外積]
$k$次微分形式$f d x_{i_1}\wedge \cdots \wedge dx_{i_k}$と$l$次微分形式$\eta = g d x_{j_1} \wedge \cdots \wedge dx_{j_l}$の外積を
$$
(f d x_{i_1}\wedge \cdots \wedge dx_{i_k})
\wedge
( g d x_{j_1} \wedge \cdots \wedge dx_{j_l})
:=
 f g \cdot  d x_{i_1} \wedge \cdots \wedge dx_{i_k}\wedge d x_{j_1} \wedge \cdots \wedge dx_{j_l}
$$
と定義する. 有限和の場合には双線形になるように定義する. つまり下が成り立つ.
$$
(\omega_1 + \omega_2) \wedge \eta = \omega_1  \wedge \eta  + \omega_2 \wedge \eta
\quad 
 \omega \wedge (\eta_1 + \eta_2) = \omega  \wedge \eta_1 + \omega \wedge \eta_2
$$

    \end{dfn}
    \end{tcolorbox}
    
  \begin{tcolorbox}[
    colback = white,
    colframe = green!35!black,
    fonttitle = \bfseries,
    breakable = true]
\begin{dfn}[外微分]
%$k=1, \ldots, m=\dim M$となる自然数とする. 
%座標近傍$(U, x_1, \ldots, x_m)$とする
$k$次微分形式$f d x_{i_1}\wedge \cdots \wedge dx_{i_k}$について, 

$$
d \left(  f d x_{i_1}\wedge \cdots \wedge dx_{i_k} \right)
= \left(\sum_{j=1}^{m}\pdrv{f}{x_{j}}d x_{j}\right)  d x_{i_1}\wedge \cdots \wedge dx_{i_k}
$$
と定義する. 有限和の場合には$\R$線形になるように定義する. 特に$k=0$のときについては以下が成り立つ. 
$$
df =  \sum_{j=1}^{m}\pdrv{f}{x_{j}}d x_{j}
$$
%\begin{align*}
%\begin{split}
%d\omega
%&=
% \sum_{ 1 \le i_1< \cdots < i_k\le m }
% df_{i_1 \cdots i_k}d x_{i_1} \wedge \cdots \wedge dx_{i_k}\\
%& =  \sum_{ 1 \le i_1< \cdots < i_k\le m }\left(\sum_{j=1}^{m}\pdrv{f_{i_1 \cdots i_k}}{x_{j}}d x_{j}\right)\wedge d x_{i_1} \wedge \cdots \wedge dx_{i_k}
%\text{となる. }
%\end{split}
%\end{align*}
    \end{dfn}
    \end{tcolorbox}
      \begin{tcolorbox}[
    colback = white,
    colframe = green!35!black,
    fonttitle = \bfseries,
    breakable = true]
\begin{dfn}[引き戻し]


 $\varphi : U \rightarrow \R^n$を
 $$
 \begin{array}{ccccc}
\varphi: &U& \rightarrow & \R^n& \\
&(x_{1}, \ldots, x_m) & \longmapsto & 
(\varphi_1 (x) , \varphi_2 (x) , \ldots, \varphi_n (x)) &
\end{array}
$$
 となる$C^{\infty}$写像とする.
 $\R^n$の開集合$V$上の$l$次微分形式$\eta = g d y_{j_1} \wedge \cdots \wedge dy_{j_l}$について, $\eta$の$\varphi$による引き戻し$\varphi^{*}\eta$を
 $$
\varphi^{*}\eta:= 
(g\circ \varphi )d \varphi_{j_1} \wedge \cdots \wedge d \varphi_{j_l} 
=
(g\circ \varphi )
\left(\sum_{i_1 =1}^{m}\pdrv{ \varphi_{j_1}}{x_{i_1}} dx_{i_1} \right)\wedge \cdots \wedge 
\left(\sum_{i_l =1}^{m}\pdrv{\varphi_{j_l}}{x_{i_l}} dx_{i_l}\right)
$$
とする. これは$M$上の$l$次微分形式となる. 有限和の場合には$\R$線形になるように定義する. 

    \end{dfn}
    \end{tcolorbox}
 
 %\begin{rem}上の定義は局所座標$(U, x_1, \ldots, x_m)$を用いていない定義である. 局所座標を用いない定義はわかりずらい印象がある.(外微分は特に局所座標の方がわかりやすいと思う). ただ証明などではこちらが便利な時もある. \end{rem}
 
 \begin{exa}

 $\omega=f_1dx_1+ f_2dx_2$,  $\eta=g_1dx_1+ g_2dx_2$, $\varphi (x_1, x_2) = (\varphi_1(x_1,x_2), \varphi_2(x_1,x_2))$とすると外積, 外微分, 引き戻しはそれぞれ次の通りとなる. 
    \begin{itemize}
    \setlength{\parskip}{0cm}
  \setlength{\itemsep}{2pt} 
\item  $
 \omega \wedge \eta = (f_1dx_1+ f_2dx_2) \wedge (g_1dx_1+ g_2dx_2)
 = (f_1g_2)dx_1 \wedge dx_2 + (f_2g_1)dx_2 \wedge dx_1 = (f_1g_2 - f_2 g_1) dx_1 \wedge dx_2.
 $
\item  $
 d \omega = \left(\pdrv{f_1}{x_1} dx_1+ \pdrv{f_1}{x_2} dx_2\right) \wedge dx_1 + \left(\pdrv{f_2}{x_1} dx_1+ \pdrv{f_2}{x_2} dx_2\right) \wedge dx_2
 %= \pdrv{f_1}{x_2} dx_2\wedge dx_1 +  \pdrv{f_2}{x_1} dx_1\wedge dx_2
= \left( -\pdrv{f_1}{x_2} + \pdrv{f_2}{x_1}\right)dx_1 \wedge dx_2.
  $
  \item
  $
 \varphi^{*}\omega
=
 f_{1} (\varphi(x)) d\varphi_{1} +  f_{2} (\varphi(x)) d\varphi_{2}
 =
  f_{1} (\varphi(x)) \left(\pdrv{\varphi_1}{x_1} dx_1 + \pdrv{\varphi_1}{x_2} dx_2 \right) 
  +  f_{2} (\varphi(x)) \left(\pdrv{\varphi_2}{x_1} dx_1 + \pdrv{\varphi_2}{x_2} dx_2 \right). 
  %&=(f_{1} (\varphi(z)) \pdrv{\varphi_1}{z_1} +  f_{2} (\varphi(z)) d\varphi_{2} (\pdrv{\varphi_2}{z_1})dz_1+ (f_{1} (\varphi(z))\pdrv{\varphi_1}{z_2}+  f_{2} (\varphi(z)) \pdrv{\varphi_2}{z_2} )dz_2
$
 \end{itemize}
 
 \end{exa}


  \begin{tcolorbox}[
    colback = white,
    colframe = green!35!black,
    fonttitle = \bfseries,
    breakable = true]
\begin{prop}
$\omega$を$k$次微分形式, $\eta$を$l$次微分形式, $\zeta$を$s$次微分形式とする. 次が成り立つ.
\begin{itemize}
 \setlength{\parskip}{0cm}
  \setlength{\itemsep}{2pt} 
\item$\omega \wedge \eta = (-1)^{kl} \eta \wedge \omega$, $\omega \wedge (\eta  \wedge \zeta)= (\omega \wedge \eta)  \wedge \zeta$. 
\item $\varphi^{*}(\omega \wedge \eta) = \varphi^{*}(\omega) \wedge \varphi^{*}(\eta)$.
\item $d(\omega \wedge \eta ) = (d \omega) \wedge \eta + (-1)^{k}\omega \wedge (d \eta)$.
    %$M$上の$C^{\infty}$級ベクトル場の集合を$\mathscr{X}(M)$で表す. 
    \item $d(d \omega)=0$, $d(\varphi^{*}\omega)=\varphi^{*}(d \omega)$.
\end{itemize}
    \end{prop}
    \end{tcolorbox}
  
  \newpage 


以下の問題に答えよ. 
ただし関数の定義域などに関しては"うまく"取るものとする.\footnote{例えば$\frac{1}{x^2 + y^2} dx $については$\R^2\setminus \{ 0\}$上で考えるものとする. }
\vspace{11pt}

\begin{enumerate}[label=\textbf{問}\ref*{sec-Rn-diff}.\arabic*]

\item $^{\bullet}$ $f(x,y,z)=x^2 + y^2$, $g(x,y,z) = xyz$について, $df$と$dg$を求めよ.

\item $^{\bullet}$ $f(r, \theta) = e^{-r^2} \cos \theta$, $g(x,y,z) = \log (x^2 + y^2 + z^2)$について, $df$と$dg$を求めよ.

\item $^{\bullet}$$(x dx + y dy ) \wedge (-x dx + y dy)$と$(x dx + y dy ) \wedge (-y dx +  x dy)$を計算せよ. 

\item $^{\bullet}$ $\omega = \sum_{i=1}^{m} f_i dx_i$, $\eta= \sum_{j=1}^{m} g_j dx_j$について, $\omega \wedge \eta$を計算せよ.  

\item $^{\bullet}$ $(xdx + y dy) \wedge (ydy + zdz) \wedge (xdx + zdz)$を計算せよ. 

%\item $^{\bullet}$ $d(r \cos \theta) \wedge d(r \sin \theta)$を計算せよ.

\item $^{\bullet}$ $\omega = dz - y dx$, $\eta = \cos z dx +  \sin z dy$について, $d \omega$と$d \eta$をそれぞれ求めよ. 

\item $^{\bullet}$ $\omega = \frac{-y}{x^2 + y^2} dx + \frac{x}{x^2 + y^2} dy$について, $d \omega$を求めよ. 

\item $^{\bullet}$  $n$変数$C^{\infty}$級関数$f$について$d (df)=0$を(計算によって)示せ.

\item $^{\bullet}$ $\varphi(x, y) = (x^m, y^n)$とし, $\eta = \frac{1}{x}dx + dy$とする. 
$\varphi^{*}\eta$を求めよ. 

\item $^{\bullet}$ $\varphi(r, \theta) = (r \cos \theta, r \sin \theta)$とし, $\eta = \frac{-y}{x^2 + y^2} dx + \frac{x}{x^2 + y^2} dy$とする. 
$\varphi^{*}\eta$を求めよ. 

\item $^{\bullet}$ $\varphi(x, y) = (x + y^2, 2y)$とし, $\eta = dx \wedge dy$とする. 
$\varphi^{*}\eta$を求めよ. 

\item $^{\bullet}$ $\varphi(r, \theta) = (r \cos \theta, r \sin \theta)$とし, $\eta = \frac{1}{x^2 + y^2} dx \wedge dy$とする. 
$\varphi^{*}\eta$を求めよ. 

\item $\varphi( \theta, \rho) = ( \sin \theta \cos \rho, \sin \theta \sin \rho,  \cos \theta)$とし, $\eta = z dx \wedge dy  + y dz \wedge dx + x dy \wedge dz$とする. $\varphi^{*}\eta$を求めよ. 


\item $\varphi(r, \theta, \rho) = (r  \sin \theta \cos \rho, r  \sin \theta \sin \rho, r  \cos \theta)$とし, $\eta = dx \wedge dy \wedge dz$とする. $\varphi^{*}\eta$を求めよ. 

\item $\R^{2n}$上の2次微分形式$\omega = \sum_{i=1}^{n} dx_{2i-1} \wedge dx_{2i}$について$\omega^n$を求めよ.

\item $^{*}$ $X = \R^3 \setminus \{(0,0,0)\}$とし, $f(x,y,z)$を$X$上の$C^{\infty}$級関数で$r = \sqrt{x^2 + y^2 + z^2}$を用いて
$f(x,y,z) = h(r)$とかけているとする.
$X$上の1次微分形式$\omega$を
$$
\omega = f(x,y,z)(x dx + y dy + z dz)
$$
とする. 次の問いに答えよ.
\begin{enumerate}
 \setlength{\parskip}{0cm}
  \setlength{\itemsep}{2pt} 
\item $d\omega =0$を示せ. (このとき$\omega$は閉形式であるという. )
\item ある$C^{\infty}$級関数$g$があって$\omega =dg$となることを示せ. (このとき$\omega$は完全形式であるという.)
\item $\Delta \varphi=0$となる$C^{\infty}$級関数$\varphi$によって$\omega =d\varphi$となるとき, $f$を$x,y,z$を用いて表せ. 
ここで
$$
\Delta \varphi=\pdrv{^2\varphi}{x^{2}}+\pdrv{^2\varphi}{y^{2}}+\pdrv{^2\varphi}{z^{2}}
\text{である.}
$$
\end{enumerate}
\end{enumerate}


\newpage
\begin{center}
\section{$\R^n$上の微分形式の応用問題 -線積分とド・ラーム コホモロジー-}
\label{sec-Rn-diff-ap}
\end{center}
\begin{flushright}
 岩井雅崇(いわいまさたか)
\end{flushright}

一部の内容は授業の後半の内容も含む.  そのため授業の後半でこれらの問題を解いても良い. 


 \begin{tcolorbox}[
    colback = white,
    colframe = green!35!black,
    fonttitle = \bfseries,
    breakable = true]
\begin{dfn}
$U$を$\R^n$の開集合とし, $\gamma : [a,b] \to U$を$C^\infty$曲線とする. 
$U$上の$1$次微分形式$\omega = f_1 dx_1 + \cdots + f_n dx_n$について, $\omega$の$\gamma$に沿った線積分を
    $$
\int_{\gamma} \omega = \int_{a}^{b} \sum_{i=1}^{n} f_i (\gamma (t)) \drv{\gamma_i}{t} dt
    $$
と定義する. ここで$\gamma(t) = (\gamma_1 (t), \ldots, \gamma_n(t))$であるとする. 
   \end{dfn}
    \end{tcolorbox}
    
 \begin{tcolorbox}[
    colback = white,
    colframe = green!35!black,
    fonttitle = \bfseries,
    breakable = true]
\begin{thm}[ポアンカレの補題 (Poincareの補題)]
$k$を1以上の整数とする. $\R^n$上の$k$次微分形式が$d \omega=0$ならば, ある$k-1$次微分形式$\eta$があって, $\omega = d \eta$とかける.  
   \end{thm}
    \end{tcolorbox}
\begin{rem}
$U$上の微分形式$\omega$について, 「$\omega = d\eta$ならば$d \omega=0$」は常に正しい. しかし逆は成り立たない. またポアンカレの補題は$U$が星型\footnote{$U$が星型であるとは, ある点$p \in U$があって, 任意の点$x$と任意の$t \in [0,1]$について$(1-t)p + t x \in U$が成り立つこと. 例えば開円板は星型だが, $\R^2 \setminus \{ 0\}$は星型ではない.}でも成り立つ. 
\end{rem}
 
 一足早いがド・ラーム コホモロジーを定義する. (なお授業後半の内容のため, 現時点で理解する必要はない.)
 
 \begin{tcolorbox}[
    colback = white,
    colframe = green!35!black,
    fonttitle = \bfseries,
    breakable = true]
\begin{dfn}[ド・ラーム コホモロジー(de Rham コホモロジー)]
$U$を$\R^n$の開集合, $k$を0以上の整数, $\omega$を$k$次微分形式とする. 
\begin{itemize}
 \setlength{\parskip}{0cm}
  \setlength{\itemsep}{2pt}
\item  $d\omega=0$なる微分形式を\underline{閉形式}という. $k$次微分形式で閉形式であるものからなるベクトル空間を$Z^k(U)$と書く. 
\item ある$k-1$次微分形式$\eta$があって$\omega = d \eta$とかけるとき, $\omega$は\underline{完全形式}と呼ばれる. $k$次微分形式で完全形式であるものからなるベクトル空間を$B^k(U)$と書く. このとき$B^k(U) \subset Z^k (U)$である. つまり完全形式は閉形式である. 
\item $H^{k}_{DR}(U) := Z^k (U)/B^k(U)$とし, \underline{$U$の$k$次de Rhamコホモロジー}という
 \end{itemize}
\end{dfn}
\end{tcolorbox}  

ド・ラーム コホモロジーを使うとポアンカレの補題は以下のようにかける. 
 \begin{tcolorbox}[
    colback = white,
    colframe = green!35!black,
    fonttitle = \bfseries,
    breakable = true]
\begin{thm}[ポアンカレの補題]
1以上の整数$k$について$H^{k}_{DR}(\R^n) =0$.
   \end{thm}
    \end{tcolorbox}
同様に$\R^n$の開集合$U$が星型ならば, 1以上の整数$k$について$H^{k}_{DR}(U) =0$である. 
\vspace{11pt}
\begin{enumerate}[label=\textbf{問}\ref*{sec-Rn-diff-ap}.\arabic*]


%\item $V$を$\R$上の$m$次元ベクトル空間とし, $\{ e_1, \ldots, e_m\}$を$V$の基底とし, $\{ \omega_1, \ldots, \omega_m\}$は$V^{*}$を $\{ e_1, \ldots, e_m\}$の双対基底とする. また$k$を2以上の自然数とする. 次の問いに答えよ
%\begin{enumerate}
% \setlength{\parskip}{0cm}
%  \setlength{\itemsep}{2pt} 
%\item $k$次多重線型形式のなす空間$\otimes^{k} V^{*}$の基底を一つ構成せよ. また$\otimes^{k} V^{*}$の次元を求めよ.
%\item $k$次対称形式のなす空間${\rm S}^{k}(V^{*})$の基底を一つ構成せよ. また${\rm S}^{k}(V^{*})$の次元を求めよ.
%\item $k$次交代形式のなす空間$\wedge^{k} V^{*}$の基底を一つ構成せよ. また$\wedge^{k} V^{*}$の次元を求めよ.
%\end{enumerate}

\newpage 

%\item $^{\bullet}$ $\R^m$上の$C^\infty$級関数$f$について, $df=0$ならば$f$は定数関数であることをしめせ. 

\item $^{\bullet}$ $\gamma : [0, 1] \to \R^{2}$を$\gamma(t) = (t^2,  t)$とする. 
線積分$\int_{\gamma} xdx - ydy$を計算せよ.

\item $^{\bullet}$ $U$を$\R^n$の開集合とし, $f$を$U$上の$C^\infty$級関数とする. 
$\gamma : [a,b] \to U$なる$C^\infty$曲線に関して, 
$$
\int_{\gamma} df = f(\gamma(b)) - f(\gamma (a))
$$
であることを示せ. 


\item $^{\bullet}$ $\R^{2} \setminus \{0\}$上の1次微分形式
$$
\omega = \frac{-ydx + x dy}{x^2+y^2}
$$
について次の問いに答えよ.
\begin{enumerate}
 \setlength{\parskip}{0cm}
  \setlength{\itemsep}{2pt} 
%\item 極座標$(x,y)=(r \cos \theta, r \sin \theta)$を用いて$\omega$を$dr,d\theta$で表せ. 
\item $d \omega=0$を示せ. つまり$\omega$は閉形式である. 
\item $\gamma : [0, 2\pi] \to \R^{2} \setminus \{0\} $を$\gamma(t) = (\cos t, \sin t)$とする. 
線積分$\int_{\gamma} \omega $を計算せよ. 
\item $\omega = d g$となる$C^{\infty}$級関数は存在しないことを示せ. つまり$\omega$は完全形式ではない. 
\end{enumerate}


\item 
 $\R^3$の関数(スカラー場)$F(x,y,z)$とベクトル場${\bf V}(x,y,z) = (V_1(x,y,z), V_2(x,y,z), V_3(x,y,z))$について, 
$$
{\rm grad}(F)=\nabla F=\left(\pdrv{F}{x}, \pdrv{F}{y}, \pdrv{F}{z}\right) \quad {\rm div}({\bf V}) = \nabla \cdot {\bf V} = \pdrv{V_1}{x}+\pdrv{V_2}{y}+ \pdrv{V_3}{z}
$$
$$
{\rm rot}({\bf V})=\nabla \times {\bf V}
=\left(\pdrv{V_3}{y}- \pdrv{V_2}{z}, \pdrv{V_1}{z} - \pdrv{V_3}{x}, \pdrv{V_2}{x} - \pdrv{V_1}{y}\right)
$$
%$${\rm div}({\bf V}) = \nabla \cdot {\bf V} = \pdrv{V_1}{x}+\pdrv{V_2}{y}+ \pdrv{V_3}{z}$$
と定義する. 次の問いに答えよ.\footnote{この問題は「$\R^3$上のベクトル解析が微分形式によって再解釈される」ことを確かめる問題である.そのため数学的な記述は少々曖昧であるのでご了承いただきたい.}
\begin{enumerate}
 \setlength{\parskip}{0cm}
  \setlength{\itemsep}{2pt} 
\item 下の図式が可換になるように$\Phi_1,\Phi_2, \Phi_3$をうまく定義せよ. %次の図式を考える.
\begin{equation*}
\hspace{-55pt}
\xymatrix@C=30pt@R=20pt{
\{\text{関数(スカラー場)}\}\ar@{=}[d]\ar@{->}[r]^{\hspace{15pt}{\rm grad}} &\{\text{ベクトル場}\}\ar@{->}[r]^{{\rm rot}}  \ar@{->}[d]^{\Phi_1}
&\{\text{ベクトル場} \}\ar@{->}[d]^{\Phi_2}\ar@{->}[r]^{{\rm div}\hspace{15pt}} &\{\text{関数(スカラー場)}\}  \ar@{->}[d]^{\Phi_3}\\ 
\{\text{関数(0次微分形式)}\}\ar@{->}[r]^{\hspace{15pt} d}&\{\text{1次微分形式}\} \ar@{->}[r]^{d}
&\{ \text{2次微分形式}\}\ar@{->}[r]^{d}&\{ \text{3次微分形式}\}\\
 }
\end{equation*}
%上の図式が可換になるようにうまく$\Phi_1,\Phi_2, \Phi_3$を定義せよ.
\item ${\rm rot}({\rm grad}(F)) =0$と${\rm div}({\rm rot}({\bf V}) =0$をそれぞれ示せ.
\item $\R^3$のベクトル場${\bf V}$について, ${\rm rot}{\bf V}\equiv 0$であることは${\bf V} = {\rm grad}\phi$なるスカラー場(スカラー・ポテンシャル)$\phi$が存在することと同値であることを示せ.\footnote{ヒント: ポアンカレの補題. 同様に${\rm div}{\bf V}\equiv 0$であることは${\bf V} = {\rm rot}{\bf A}$なるベクトル場(ベクトル・ポテンシャル)${\bf A}$が存在することと同値であることがわかる. }
%\item ${\rm div}{\bf V}\equiv 0$であることは${\bf V} = {\rm rot}{\bf A}$なるベクトル場(ベクトル・ポテンシャル)${\bf A}$が存在することと同値であることを示せ. 
\end{enumerate}

\newpage

\item $^{}$ [Tu. Problem 19.13] 次を英訳し問題に解答せよ.
%[Tu. Problem 19.13 Twentieth-century formulation of Maxwell’s equations] 次を英訳し問題に解答せよ.

In Maxwell's theory of electricity and magnetism, developed in the late nineteenth century,
the electric field ${\bf E} = (E_1, E_2, E_3)$ and the magnetic field ${\bf B} = (B_1, B_2, B_3)$ in a vacuum
$\R^3$
with no charge or current satisfy the following equations:
$$
{\rm rot}{\bf E} = - \pdrv{{\bf B}}{t} ,  
{\rm rot} {\bf B} = \pdrv{{\bf E}}{t}, 
{\rm div} {\bf E} = 0,
{\rm div} {\bf B} = 0. 
$$
We define the 1-form $E$ on $\R^3$ corresponding to the vector field $ { \bf E}$ by $E = E_1 dx + E_2 dy + E_3 dz$ and define the 2-form $B$ on $\R^3$ corresponding to the vector field ${ \bf B}$ by $B = B_1 dy \wedge dz + B_2 dz \wedge dx + B_3 dx \wedge dy$. 
%By the correspondence in Subsection 4.6, the 1-form $E$ on $\R^3$ corresponding to the vector field $ { \bf E}$ is $E = E_1 dx + E_2 dy + E_3 dz$ and the 2-form $B$ on $\R^3$ corresponding to the vector field ${ \bf B}$ is $B = B_1 dy \wedge dz + B_2 dz \wedge dx + B_3 dx \wedge dy$. 

Let $\R^4$ be space-time with coordinates $(x, y, z, t)$. 
Then both $E$ and $B$ can be viewed as
differential forms on $\R^4$. Define $F$ to be the 2-form
$$
F = E \wedge dt + B
$$
on space-time. Decide which two of Maxwell's equations are equivalent to the equation $dF =0$.
Prove your answer. \footnote{この文章には続きがあった. "The other two are equivalent to $d * F = 0$ for a star-operator $*$ defined indifferential geometry."つまり後二つは$d * F =0$と同じである. ここで$*$はHodge-star operatorである. }

\item \label{poincare} 以下は$\R^2$におけるPoincareの補題に関する問題である. 次の問いに答えよ.
\begin{enumerate}
\setlength{\parskip}{0cm}
  \setlength{\itemsep}{2pt} 
\item $\omega$を$d \omega=0$となる$\R^2$の$1$次微分形式とする.
$(x,y) \in \R^2$について$L_{(x,y)}$を0が始点で$(x,y) $が終点となる線分とし, 
$$g(x,y) = \int_{L_{(x,y) }} \omega \quad (x \in \R^2)$$
とおく. このとき$g(x,y)$は$\omega=dg$となる$\R^2$上の$C^{\infty}$級関数であることを示せ. 
\item 上と同様にして$\R^2$の$2$次微分形式$\eta$についてある$1$次微分形式$\omega$があって$\eta = d \omega$となることを示せ. 
%\item $\R^2$のド・ラームコホモロジー群$H^{k}_{DR}(\R^2)$($k=0,1,2,\ldots$)を求めよ. \footnote{余裕があれば$H^{k}_{DR}(\R^n)$はどうなるか考察せよ. }
\end{enumerate}

\end{enumerate}

\newpage


\begin{center}
\section{多様体上の微分形式}
\label{sec-mfd-diff}
\end{center}
\begin{flushright}
 岩井雅崇(いわいまさたか)
\end{flushright}

    \begin{tcolorbox}[
    colback = white,
    colframe = green!35!black,
    fonttitle = \bfseries,
    breakable = true]
\begin{dfn}

\begin{itemize}
 \setlength{\parskip}{0cm}
  \setlength{\itemsep}{2pt} 
\item $p \in M$について, 接ベクトル空間$T_{p}M$の双対空間を\underline{余接ベクトル空間}と呼び$T_{p}^{*}M$と表す.
\item 任意の$p \in M$について$\omega_{p} \in T_{p}^{*}M$が一つずつ対応しているとき, その対応$\omega = \{ \omega_p\}_{p \in M}$を\underline{$M$上の1次微分形式}という.
\item 座標近傍$(U, x_1, \ldots, x_m)$について$(dx_{i})_{p}$を
    $$
     \begin{matrix}
    (dx_i )_{p} : &T_{p}M & \rightarrow &\R\\
    & a_1\left( \pdrv{}{x_1}\right)_p  + \cdots + a_m\left(\pdrv{}{x_m}\right)_p & \mapsto  &a_i
    \end{matrix}
     $$
    とし, $U$上の1次微分形式$dx_i  := \{ (dx_{i})_p\}_{p \in U}$と定義する. これにより$M$上の1次微分形式は座標近傍$(U, x_1, \ldots, x_m)$について, ある$U$上の関数$f_i : U \rightarrow \R$があって
    $$
    \omega|_{U} = f_1dx_1 + \cdots + f_mdx_m
    $$
    とかける. 各座標近傍$(U, x_1, \ldots, x_m)$について上の$f_i $が$C^{\infty}$級となるとき, $\omega$は\underline{$C^{\infty}$級1次微分形式}という.
    %$M$上の$C^{\infty}$級ベクトル場の集合を$\mathscr{X}(M)$で表す. 
\end{itemize}
    \end{dfn}
    \end{tcolorbox}
    
     \begin{tcolorbox}[
    colback = white,
    colframe = green!35!black,
    fonttitle = \bfseries,
    breakable = true]
\begin{dfn}
$k$を0以上の整数とする. 任意の$p \in M$について$\omega_{p} \in \wedge^{k} T_{p}^{*}M$が一つずつ対応しているとき, その対応$\omega = \{ \omega_p\}_{p \in M}$を\underline{$M$上の$k$次微分形式}という.
$M$上の$k$次微分形式$\omega$は座標近傍$(U, x_1, \ldots, x_m)$について, ある$U$上の関数$f_{i_1 i_2 \cdots i_k}: U \rightarrow \R$($1 \le i_1< \cdots < i_k \le m$)があって
    $$
    \omega|_{U} = \sum_{ 1 \le i_1< \cdots < i_k\le m }f_{i_1 i_2 \cdots i_k}d x_{i_1}\wedge dx_{i_2} \wedge \cdots \wedge dx_{i_k}
    $$
    とかける. 各座標近傍$(U, x_1, \ldots, x_m)$について上の$f_{i_1 i_2 \cdots i_k}$が$C^{\infty}$級となるとき, $\omega$は\underline{$C^{\infty}$級$k$次微分形式}であるという.
    \end{dfn}
    \end{tcolorbox}

外積・外微分・引き戻しも以下のように定義される. 

 \begin{tcolorbox}[
    colback = white,
    colframe = green!35!black,
    fonttitle = \bfseries,
    breakable = true]
\begin{dfn}[外積]
%$k=1, \ldots, m=\dim M$となる自然数とする. 

 \setlength{\parskip}{0cm}
  \setlength{\itemsep}{2pt} 
  $M$上の$k$次微分形式$\omega$と$l$次微分形式$\eta$について, その\underline{外積$\omega \wedge \eta $}を
$$
\omega \wedge \eta(X_1, \ldots, X_{k+l})
=
\frac{1}{k! l!} \sum_{\sigma \in S_{k+l}} \sgn(\sigma) \omega(X_{\sigma(1)}, \ldots, X_{\sigma(k)}) \eta(X_{\sigma(k+1)}, \ldots, X_{\sigma(k+l)})
\text{とする.}
$$
    \end{dfn}
    \end{tcolorbox}
    
  \begin{tcolorbox}[
    colback = white,
    colframe = green!35!black,
    fonttitle = \bfseries,
    breakable = true]
\begin{dfn}[外微分]
%$k=1, \ldots, m=\dim M$となる自然数とする. 
%座標近傍$(U, x_1, \ldots, x_m)$とする
$M$上の$k$次微分形式$\omega$について, \underline{外微分$d \omega$}を
\begin{align*}
\begin{split}
d\omega(X_1, \ldots, X_{k+1})
&=
\sum_{i=1}^{k+1}(-1)^{i+1}X_i(\omega(X_1, \ldots,\widehat{X_{i}}, \ldots, X_{m}))\\
&+\sum_{i<j}(-1)^{i+j}\omega([X_i, X_j], X_1, \ldots, \widehat{X_{i}},  \ldots, \widehat{X_{j}}, \ldots, X_{m}).
%\text{とする.}
\end{split}
\end{align*}
とする.  ここで$(X_1, \ldots, X_{k+1})$はベクトル場とし, $(X_1, \ldots,\widehat{X_{i}}, \ldots, X_{m})$は$(X_1, \ldots,X_{i-1}, X_{i+1}, \ldots, X_{m})$を意味する.

    \end{dfn}
    \end{tcolorbox}
      \begin{tcolorbox}[
    colback = white,
    colframe = green!35!black,
    fonttitle = \bfseries,
    breakable = true]
\begin{dfn}[引き戻し]
%$k=1, \ldots, m=\dim M$となる自然数とする. 

 $\varphi : M \rightarrow N$を$C^{\infty}$写像とする. $N$上の$l$次微分形式$\eta$について, $\eta$の$\varphi$による\underline{引き戻し$\varphi^{*}\eta$}を
$$
(\varphi^{*}\eta)_{p}(X_{p}) := \eta_{\varphi(p)}((d\varphi)_{p} X_{p}) \quad (\forall p \in M, \forall X \in T_{p}M)
$$
と定める. これは$M$上の$l$次微分形式となる. 
    \end{dfn}
    \end{tcolorbox}
    
    \begin{rem}
    重要なこととして, これらは座標近傍$(U, x_1, \ldots, x_m)$をとってしまえば$\R^m$の開集合上で定義したものと同じになる! 上の定義は座標によらないというメリットがある一方でわかりずらいというデメリットもある. 
    \end{rem}
    

%問題の上に$^{\bullet}$がついている問題は\underline{解けてほしい}問題である. 問題の上に$^{*}$がついている問題は\underline{面白いかちょっと難しい}問題である. 

%以下断りがなければ$M,N$は$C^{\infty}$級多様体とし, $m = \dim M$とする.$\R^n$をユークリッド空間とし, $S^n \subset \R^{n+1}$を半径1の$n$次元球面とする.

\vspace{11pt}
\begin{enumerate}[label=\textbf{問}\ref*{sec-mfd-diff}.\arabic*]


\item \label{circle} $^{\bullet}$ $S^1 \subset \R^2$を円周とする. 
$$
U = S^1 \setminus \{(1,0)\} \quad \varphi_U(\cos\theta_U , \sin \theta_U) =\theta_U \quad (0 < \theta_U < 2\pi)
$$
$$
V= S^1 \setminus \{(-1,0)\} \quad \varphi_V(\cos\theta_V, \sin \theta_V) =\theta_V \quad (-\pi < \theta_V < \pi)
$$
として座標近傍$(U, \varphi_U), (V, \varphi_V)$を定める.
$U$上の1次微分形式$\alpha_U$と$V$上の1次微分形式$\alpha_V$を
$$
\alpha_U = d \theta_U, \quad \alpha_U = d \theta_V 
$$
とする.\footnote{厳密には$d \theta_U$は$\varphi(U)$上の微分形式と同一視している.} 
このとき$U \cap V$上で$\alpha_U = \alpha_V$であることを示せ. 
これにより$S^1$上の微分形式$\alpha$を
$$
\alpha_x = \left\{
\begin{array}{ll}
(\alpha_U)_x  & (x \in  U)\\
(\alpha_V)_x & (x \in V)
\end{array}
\right.
$$
として定めることができる. 

\item $^{\bullet}$ $\varphi: S^1 \to \R^2$を包含写像とし, $\alpha$を\ref{circle}での$S^1$上の1次微分形式であるとする. このとき以下が成り立つことを示せ. 
$$
\varphi^{*}\left( \frac{-ydx + x dy}{x^2+y^2} \right) = \alpha
$$

\item \label{complax_plane} $^{\bullet}$(多様体の基礎 20章) リーマン球面$\C\mathbb{P}^1 = \C \cup \C$を構成する2つの複素平面$\C$をそれぞれ$z = x + iy$, $\xi = \zeta + i\eta$の対応で$(\zeta, \eta)$平面, $(x,y)$平面と同一視する. 次の問いに答えよ.
\begin{enumerate}
 \setlength{\parskip}{0cm}
  \setlength{\itemsep}{2pt} 
\item 座標変換$z= \frac{1}{\xi}$を$(\zeta, \eta)$と$(x,y)$を用いて表せ.
\item $(x,y)$平面上の2次微分形式$\omega = \frac{dx \wedge dy}{(1+x^2+y^2)^2}$とする. $\omega$を$(\zeta, \eta)$を用いて表せ.  $(\zeta, \eta)$を用いて表されたものを$\omega'$とする.
\item $\omega'$は$(\zeta, \eta)$平面上の2次微分形式であることを示せ. 
\item $\omega$は$\C\mathbb{P}^1$上の2次微分形式$\widetilde{\omega}$に拡張できることを示せ.
%\footnote{ヒント: $(\zeta, \eta)$平面上の$C^{\infty}$級2次微分形式$\alpha$で, $(\zeta, \eta)$平面と$(x,y)$平面の共通部分で$\omega$と一致するものを一つ見つけよ. そうすると$\alpha$と$\omega$の貼り合わせで$\widetilde{\omega}$が構成できる.}\item $\int_{\C\mathbb{P}^1 } \widetilde{\omega}$の値を求めよ.
\end{enumerate}

\item $^{\bullet}$ $i : S^2 \rightarrow \R^3$を包含写像とする. 次の問いに答えよ.
\begin{enumerate}
 \setlength{\parskip}{0cm}
  \setlength{\itemsep}{2pt} 
\item $i^{*}(dx \wedge dy \wedge dz)$を求めよ.
\item $B = \{ (u,v) \in \R^2 | u^2 + v^2 < 1\}$とし, $\varphi : B \to S^2$を$\varphi(u,v)= (u, \sqrt{1 - u^2 - v^2}, v)$とする. $(i \circ \varphi)^{*}(dx \wedge dy)$の値が0になる$B$の点を全て求めよ. 
\item $i^{*}(dx \wedge dy)$の値が0になる$S^2$の点を全て求めよ.
\end{enumerate}

\item $^{\bullet}$ $i : S^2 \rightarrow \R^3$を包含写像とする. $i^{*}( z dx \wedge dy  + y dz \wedge dx + x dy \wedge dz)$は$S^2$のどの点でも0にならない2次微分形式であることを示せ. 

\item \label{stereo} $^{\bullet}$ $S^2 \subset \R^3$を球面とする. 
多様体の基礎の6章のように立体射影を次のように定義する: $U : = S^2 \setminus \{ (0,0,1)\}$とし
座標近傍$(U, s, t)$を
$$
s = \frac{x}{1-z}, \quad t = \frac{y}{1-z}
$$
と定義する. 
$f : S^2 \to \R$を$f(x,y,z)=z$とし, $S^2$上の1次微分形式$\omega=df$とするとき, $\omega|_{U}$を$(s,t)$
を用いて表せ.

\item \ref{stereo}の$(U, s, t)$について, $U$上の1次微分形式
$$
\alpha = \frac{-t ds + s dt}{(1 + s^2 + t^2)^2}
$$
を考える. このとき$S^2$上の1次微分形式$\tilde{\alpha}$で$\tilde{\alpha}|_{U} = \alpha$となるものが存在することを示せ.

\item \ref{stereo}の$(U, s, t)$について, $U$上の1次微分形式
$$
\frac{-t ds + s dt}{1 + s^2 + t^2}, \quad{} \frac{s ds - t dt}{(1 + s^2 + t^2)^2}
$$
がそれぞれの$S^2$上の1次微分形式に拡張できるかどうか調べよ. 

\item $^{*}$ (Tu Exercise 19.11) $f : \R^3 \to\R$を$C^\infty$級関数とし, $0$を$f$の正則値とする.
このとき$M = f^{-1}(0)$とすると$M$は$\R^3$の2次元部分多様体となる.
$f_x, f_y, f_z$を$f$の$x,y,z$に関する偏微分とするとき,
$$
\frac{dx \wedge dy}{f_z} = \frac{dy \wedge dz}{f_x} = \frac{dz \wedge dx}{f_y}
$$
が成り立つことを示せ.\footnote{ただし$\frac{dy \wedge dz}{f_x} $は$f_x \neq 0$なるところで考える. 他も同様.} また$M$上にどの点でも消えない2次微分形式が存在することを示せ. 

\item (多様体の基礎 19章) $f : M \rightarrow \R$を$C^{\infty}$級関数とする. 微分写像$df_{p} : T_{p}M \rightarrow T_{f(p)}\R \cong \R$により, $df:= \{df_{p}\}_{p \in M}$は$M$上の微分形式だと思える.
 座標近傍$(U, x_1, \ldots, x_m)$を用いて, 微分形式$df$は次のように表せることを示せ. 
$$ df|_{U} = \pdrv{f}{x_1} dx_1 + \cdots + \pdrv{f}{x_m}dx_{m} $$ 

\item (多様体の基礎 19章) $f : M \rightarrow \R$を$C^{\infty}$級関数とする. 
%\begin{enumerate}
% \setlength{\parskip}{0cm}
%  \setlength{\itemsep}{2pt} 
$X$をベクトル場とするとき, $(df) (X) = X(f)$を示せ. 
 %\footnote{定義から求める方法と局所座標を用いて示す方法の2種類がある.}
%$p \in M$について$(df)_p (X_{p}) = X_{p}(f)$が成り立つことを示せ. \footnote{定義から求める方法と局所座標を用いて示す方法の2種類がある.}
%\end{enumerate}
%であることを示せ.

\item (多様体の基礎 20章) $\omega$を1次微分形式, $X,Y$を$M$上のベクトル場とするとき$$d \omega(X,Y) = X(\omega(Y)) - Y(\omega(X)) -\omega([X,Y])\text{を示せ.}$$

\item $\omega$を$\R^n$上の1次微分形式とし, $S_{\omega}$を$\R^n$のベクトル場$X$で$\omega(X)=0$となるものの集合とする.  $d \omega \wedge \omega =0$ならば任意の$X,Y \in S_{\omega}$について$[X,Y] \in S_{\omega}$であることを示せ.

\item \label{tm_const } $TM = \cup_{p \in M}T_{p}M = \cup_{p \in M}\{ (p,v) | v \in T_{p}M\}$とし, $\{ (U_{\lambda}, \varphi_{\lambda})=(U_{\lambda}, x_{1}^{\lambda}, \ldots, x_{m}^{\lambda})\}_{\lambda \in \Lambda}$を$M$の座標近傍系とする. $\lambda \in \Lambda$について次のように写像を定める.
$$
\begin{matrix}
\pi :& TM &\rightarrow& M& &\Phi_{\lambda} :& \pi^{-1}(U_{\lambda})& \rightarrow& \varphi_{\lambda}(U_{\lambda}) \times \R^{m} \\
	& (p,v) &\mapsto& p& &					& (p, \sum_{i=1}^{m} a_{i} \left(\pdrv{}{x_{i}^{\lambda}}\right)_p)& \rightarrow& (\varphi_{\lambda}(p), (a_1, \ldots, a_{m})) \\
\end{matrix}
$$
次の問いに答えよ. 

\begin{enumerate}
\item $\Phi_{\lambda}$は$\pi^{-1}(U_{\lambda})$と$\varphi_{\lambda}(U_{\lambda}) \times \R^{m} $の一対一対応を与えることを示せ.
%\item $TM$の位相で任意の$\lambda \in \Lambda$について$\pi^{-1}(U_{\lambda})$が開集合で$\Phi_{\lambda}$が位相同型になるようなものが存在することを示せ.
\item $TM$には$\{( \pi^{-1}(U_{\lambda}), \Phi_{\lambda} )\}_{\lambda \in \Lambda}$が座標近傍系になるような$2m$次元の$C^{\infty}$級多様体の構造が入ることを示せ.\footnote{ただし「$TM$の位相で任意の$\lambda \in \Lambda$について$\pi^{-1}(U_{\lambda})$が開集合で$\Phi_{\lambda}$が位相同型になるものがある」ことは認めて良い. } 
$(TM, \pi)$を接ベクトル束という. 
%\footnote{ベクトル束に関しては, 例えば「今野 微分幾何学」を参照のこと. 実はヤコビ行列を用いても接ベクトル束を構成することができる. }
\end{enumerate}


\item $T^{*}M = \cup_{p \in M}T_{p}^{*}M$に$2m$次元の$C^{\infty}$級多様体の構造が入ることを示せ. 同様にして1以上の自然数$k$について$\wedge^{k}T_{p}^{*}M$にも多様体の構造が入るが, その次元も求めよ. なお$T^{*}M = \cup_{p \in M}T_{p}^{*}M$を余接ベクトル束という. 

%$k$を1以上の自然数とする. $TM = \cup_{p \in M}T_{p}M$や$T^{*}M = \cup_{p \in M}T_{p}^{*}M$に$C^{\infty}$級多様体の構造が入ることを示しその次元を求めよ. 同様にして$\wedge^{k}T_{p}^{*}M$にも多様体の構造が入るが, その次元も求めよ. なお$TM = \cup_{p \in M}T_{p}M$は接ベクトル束といい$T^{*}M = \cup_{p \in M}T_{p}^{*}M$は余接ベクトル束という. 

\item $M$を$\dim M = 2m$なる多様体とする.
$M$上の2次微分形式$\omega$で$d\omega=0$かつ$\omega^m $が任意の$p \in M$で0でないとき, $(M, \omega)$をシンプレクティック多様体という. 
以下$(M, \omega)$をシンプレクティック多様体とするとき, 次の問いに答えよ.\footnote{なおこの問題には$d\omega=0$はほぼ使わない.}
\begin{enumerate}
 \setlength{\parskip}{0cm}
  \setlength{\itemsep}{2pt} 
  \item 任意の$p \in M$と任意の$0 \neq u \in T_p M$について, ある$v \in T_p M$があって$\omega_p(u,v) \neq 0$となることを示せ. 
  %$$
%\begin{array}{ccccc}
%\omega_p : &T_p M \times T_p M & \rightarrow & \R & \\
%&(u,v) & \longmapsto & \omega_{p} (u,v)&
%\end{array}
%$$
%という双線形形式は次を満たすことを示せ
  \item $p \in M$を固定する.  $\xi \in T_p M $について
 $$
 \begin{array}{ccccc}
\omega_{\xi} : &T_p M & \rightarrow & \R & \\
&v & \longmapsto & \omega_{p} (\xi,v)&
\end{array}
 $$
 という1次微分形式$\omega_{\xi}$が定まる. そこで
 $\Phi: T_p M  \rightarrow  T_{p}^{*}M $を$\Phi(\xi)=\omega_{\xi} $で定めるとき, $\Phi$は線形同型写像であることを示せ. 
 % $$
% \begin{array}{ccccc}
%\Phi: &T_p M & \rightarrow & T_{p}^{*}M & \\
%&\xi & \longmapsto & \omega_{\xi} &
%\end{array}
% $$
% という写像が定める. $\Phi$は線形同型写像であることを示せ. 
 \end{enumerate}
 

\item $2n$次元トーラス$T^{2n}=\R^{2n}/\Z^{2n}$にはシンプレクティック多様体の構造が入ることを示せ. 

\item $^{*}$ 多様体$M$の余接ベクトル束$T^{*}M$にはシンプレクティック多様体の構造が入ることを示せ. 

\end{enumerate}
\newpage


\begin{center}
\section{多様体の向き・微分形式の積分・ストークスの定理}
\label{sec-stokes}
\end{center}
\begin{flushright}
 岩井雅崇(いわいまさたか)
\end{flushright}


 \begin{tcolorbox}[
    colback = white,
    colframe = green!35!black,
    fonttitle = \bfseries,
    breakable = true]
\begin{dfn}[多様体の向き付け]
\text{}

\begin{itemize}
 \setlength{\parskip}{0cm}
  \setlength{\itemsep}{2pt} 
\item $(U, x_1, \ldots, x_m)$と$(V, y_1, \ldots, y_m)$を$U \cap V \neq \phi$となる$M$の座標近傍とする.
$(U, x_1, \ldots, x_m)$と$(V, y_1, \ldots, y_m)$が\underline{同じ向き}であるとは, $U \cap V$上で
$$
\pdrv{(y_1, \ldots, y_m)}{(x_1, \ldots, x_m)}:=\det(\left( \pdrv{y_j}{x_i} \right)_{1\le i,j \le m}) >0
\text{となることとする.}
$$
\item \underline{$M$が向きづけ可能}であるとは, $M$の座標近傍系$\{ (U_{\lambda}, x_{1}^{\lambda}, \ldots, x_{m}^{\lambda})\}_{\lambda \in \Lambda}$であって, 同じ向きになるものが存在することとする.
\end{itemize}
    \end{dfn}
    \end{tcolorbox} 
    
    
 \begin{tcolorbox}[
    colback = white,
    colframe = green!35!black,
    fonttitle = \bfseries,
    breakable = true]
\begin{thm}[1の分割]
$M$が第二可算であると仮定する.
任意の$M$の開被覆$\{U_{\lambda}\}_{\lambda \in \Lambda}$についてある可算個の$C^{\infty}$級関数$\rho_{j} : M \rightarrow \R$($j \in \N$)があって次が成り立つ
\begin{enumerate}
 \setlength{\parskip}{0cm}
  \setlength{\itemsep}{2pt} 
\item $\{ \Supp(\rho_{j})\}_{j \in \N}$は$M$の被覆であり, $p \in M$についてある$p$の開近傍$U$をとれば$U \cap \Supp(\rho_{j}) \neq \varnothing$なる$j$は有限個になる.(局所有限な被覆という.)
\item 任意の$j \in \N$についてある$\lambda_{j} \in \Lambda$があって$\Supp(\rho_{j}) \subset U_{\lambda_j}$となる. ($\{U_{\lambda}\}_{\lambda \in \Lambda}$の細分という.)
\item $0 \le \rho_j \le 1$かつ$\sum_{j \in \N} \rho_{j} \equiv 1$.
\end{enumerate}
この$\{ \rho_{j} \}_{j \in \N}$を$\{U_{\lambda}\}_{\lambda \in \Lambda}$に従属する1の分割という. \\
ここで$\rho_j$の台を${\rm Supp}(\rho) :=\overline{\{q \in M | \rho(q) \neq 0 \}}$とする. 
     \end{thm}
    \end{tcolorbox} 
\begin{rem}
上は$\sigma$コンパクトで成り立つ定理である.(第二可算な多様体は$\sigma$コンパクトであるらしい.)ただ$\sigma$コンパクトは応用上で使うか怪しいし, 多様体に第二可算を仮定することが多いので, ここでは第二可算として主張を述べた.\footnote{「トゥー 多様体」では多様体に第二可算を仮定している.}要するに1の分割は取れると思って良い. 
\end{rem}

 %この章では多様体$M$について第二可算を仮定する. 第二可算ならばsigma compact
 
  \begin{tcolorbox}[
    colback = white,
    colframe = green!35!black,
    fonttitle = \bfseries,
    breakable = true]
\begin{dfn}
\label{integral_local}
$(U, \varphi) = (U, x_1, \ldots, x_m)$を座標近傍とし, $U$上の$m$次微分形式を$\omega = f(x_1, \ldots, x_m)dx_1 \wedge \cdots \wedge dx_m$とする.
$\varphi(U)$が正方形領域$V:=[-a,a]^{m}$に含まれるとき, $\omega$の$U$上の積分を
$$
\int_{U} \omega := \int_{[-a,a]^{m}}f(x_1, \ldots, x_m)dx_1 \wedge \cdots \wedge dx_m
\text{で定義する.}
$$
    \end{dfn}
    \end{tcolorbox} 
    


  \begin{tcolorbox}[
    colback = white,
    colframe = green!35!black,
    fonttitle = \bfseries,
    breakable = true]
\begin{thm}
$M$が向きづけ可能なコンパクト$m$次元多様体とし, $\omega$を$m$次微分形式とする.
このとき同じ向きになる$M$の座標近傍系$U_1, \ldots, U_N$とそれに従属する1の分割$\rho_{1}, \ldots, \rho_{N}$があって,
$\omega$の$M$上の積分を
$$
\int_{M} \omega := \sum_{j=1}^{N} \int_{M} \rho_j \omega
$$
で定義する. この積分の値は実数値であり, 1の分割や近傍系の取り方によらない. 
    \end{thm}
    \end{tcolorbox} 

\begin{rem}
 $\rho_j \omega$は定義\ref{integral_local}の仮定を満たすため上のように積分が定義できる. 
 またこの積分の定義は理論上役に立つが計算上ではあまり役には立たない. 
%$M$がコンパクトでない場合でも1の分割が取れれば積分は定義できるが, 有限の値になるかはわからない. 
\end{rem}


    \begin{tcolorbox}[
    colback = white,
    colframe = green!35!black,
    fonttitle = \bfseries,
    breakable = true]
    \begin{dfn}[]
    $M$を第二可算ハウスドルフ空間とする. 次の条件が成り立つとき$M$は$m$次元境界つき($C^{\infty}$級)多様体と呼ばれる.
     \begin{enumerate}
     \setlength{\parskip}{0cm}
  \setlength{\itemsep}{2pt} 
     \item $M$の開被覆$M = \cup_{\lambda \in \Lambda} U_{\lambda}$と像への同相写像
     $$
     \varphi_{\lambda} : U_{\lambda} \rightarrow \mathbb{H}^n := \{ (x_1, x_2, \ldots, x_m) \in \R^m | x_1 \geqq 0\}
     \text{が存在する.}
     $$
    % 座標近傍系$\{(U_\lambda, \varphi_\lambda)\}_{\lambda \in\Lambda}$があって, $M = \cup_{\lambda \in \Lambda} U_{\lambda}$となる. 
     \item $U_\lambda \cap U_\mu \neq \phi$なる$\lambda, \mu \in \Lambda$について
    $
   \varphi_\mu\circ \varphi_{\lambda}^{-1} : \varphi_{\lambda}(U_\lambda \cap U_\mu) \rightarrow \varphi_{\mu}(U_\lambda \cap U_\mu) 
    $
    は$C^{\infty}$級写像である
     \end{enumerate}
 $\partial M := \cup_{\lambda \in \Lambda} \varphi_{\lambda}^{-1}(\{ 0\} \times \R^{m-1}) \subset $を$M$の境界と呼ぶ.
    \end{dfn}
    \end{tcolorbox}   
$M$の境界 $\partial M$は$m-1$次元多様体となる. また$M$が向きづけ可能であるとき, $\partial M$には座標近傍系$\{(U_\lambda, x_{2}^{\lambda}, \ldots, x_{m}^{\lambda})\}_{\lambda \in \Lambda}$によって向きが入る.

  \begin{tcolorbox}[
    colback = white,
    colframe = green!35!black,
    fonttitle = \bfseries,
    breakable = true]
\begin{thm}[ストークスの定理 (Stokesの定理)]
$M$を向きづけ可能なコンパクト$m$次元境界つき多様体とし, $\eta$を$m-1$次微分形式とするとき, 次が成り立つ. 
$$
\int_{M} d \eta = \int_{\partial M} \eta 
$$
    \end{thm}
    \end{tcolorbox} 
    
    \begin{tcolorbox}[
    colback = white,
    colframe = green!35!black,
    fonttitle = \bfseries,
    breakable = true]
\begin{cor}
$M$を向きづけ可能なコンパクト$m$次元多様体とし, $\eta$を$m-1$次微分形式とするとき, $\int_{M} d \eta =0 $となる.    
\end{cor}
    \end{tcolorbox} 

%問題の上に$^{\bullet}$がついている問題は\underline{解けてほしい}問題である. 問題の上に$^{*}$がついている問題は\underline{面白いかちょっと難しい}問題である. 

%以下断りがなければ$M,N$は$C^{\infty}$級多様体とし, $m = \dim M$とする.$\R^n$をユークリッド空間とし, $S^n \subset \R^{n+1}$を半径1の$n$次元球面とする.


以下断りがなければ多様体$M$には境界がないものとする. (つまり$\partial M = \varnothing$を仮定する. )

\vspace{11pt}
\begin{enumerate}[label=\textbf{問}\ref*{sec-stokes}.\arabic*]


%\item %$^{}$[Tu Problem 13.3 Smooth Urysohn Lemma] 
%Let $A$ and $B$ two disjoint closed sets in a manifold $M$. Find $C^{\infty}$ function $f$ on $M$ such that $f$ is identically 1 on $A$ and identically 0 on $B$.

%\begin{enumerate}
%\setlength{\parskip}{0cm}
  %\setlength{\itemsep}{2pt} 
 %\item Let $A$ and $B$ two disjoint closed sets in a manifold $M$. Find $C^{\infty}$ function $f$ on $M$ such that $f$ is identically 1 on $A$ and identically 0 on $B$.
%\item Let $A$ be a closed subset and $U$ an open subset of a manifold $M$. Show that there is a $C^{\infty}$function $f$ on $M$ such that $f$ is identically 1 on $A$ and $\Supp f \subset U$.
%\end{enumerate}


\item $^\bullet$ $M$を向きづけ可能なコンパクト$m$次元多様体とし, $N $を$m-1$次元の$M$の閉部分多様体とする. $\omega$を$m$次微分形式とするとき
$
\int_{M} \omega = \int_{M \setminus N} \omega
$
を示せ. (ヒント: $M$が$\R^m$の開集合のときにはどうなるか?)

\item $^\bullet$ \ref{circle}において定義した$S^1$上の1次微分形式$\alpha$について, $\int_{S^1 } \alpha$の値を求めよ.

\item $^\bullet$ \ref{complax_plane}において定義した$\C\mathbb{P}^1$上の2次微分形式$\widetilde{\omega}$について, $\int_{\C\mathbb{P}^1 } \widetilde{\omega}$の値を求めよ.

\newpage

\item $^{\bullet}$ $$\int_{S^2} x dy \wedge dz + y dz \wedge dx + z dx \wedge dy$$を求めよ. 

\item  $^{\bullet}$ $D=[a,b] \times [c, d]$とし, $f(x,y), g(x,y)$を$D$上の$C^{\infty}$級関数とする.\footnote{厳密にいうと$D$を含む開集合$U$があって, $U$上で$f(x,y), g(x,y)$は$C^{\infty}$級である}
グリーンの定理
$$
\int_{\partial D} f(x,y) dx + g(x,y) dy = \iint_{D} \left(\pdrv{g}{x}  -\pdrv{f}{y} \right) dxdy
$$
をストークスの定理を用いて示せ. ただし$\partial D$にどのような向きを入れたか明記すること.

\item $^{\bullet}$ $\R^{2} \setminus \{ (0,0)\}$上で定義された領域上で定義された関数$f(x,y) = \frac{1}{2}\log(x^2 +y^2)$を考える. $\R^{2}  \setminus  \{ (0,0)\}$上の1次微分形式を
$
\omega := \pdrv{f}{x}dy -  \pdrv{f}{y}dx
$
とする. 次の問いに答えよ.
\begin{enumerate}
 \setlength{\parskip}{0cm}
  \setlength{\itemsep}{2pt} 
\item $\R^{2} \setminus \{ (0,0)\}$上 $\Delta f = \pdrv{^2f}{x^2} + \pdrv{^2f}{y^2} =0$であることを示せ.
\item $C_1$を中心$(3,0)$で半径2の円周とし, 向きを反時計回りに入れる. $\int_{C_1} \omega$を計算せよ.
\item $C_2$を中心$(1,0)$で半径4の円周とし, 向きを反時計回りに入れる. $\int_{C_2} \omega$を計算せよ.
\end{enumerate}

\item 2次元トーラス$T^2 = \{(x,y,z,w) \in \R^4 | x^2 + y^2 = z^2 + w^2 =1\}$について
$$
\int_{T^2} yzw \, dx \wedge dz
$$
を求めよ. 

\item $\R^3 \setminus \{ (0,0,0)\}$上の2次微分形式
$$
\omega = \frac{z dx \wedge dy  + y dz \wedge dx + x dy \wedge dz}{(x^2 + y^2+ z^2)^{\frac{3}{2}}}
$$
とする. $X$を$\R^3$内の有界な境界つき3次元多様体で$(0,0,0) \not \in \partial X $であるものとする. 
この$X$について
$$
\Omega = \int_{\partial X} \omega
$$
と定める. このとき$(0,0,0) \in X$ならば$\Omega=4 \pi$であり, $(0,0,0) \not \in X$ならば$\Omega=0$であることを示せ.  

\newpage
\item \label{nform} (多様体の基礎 20章)
 $M$を多様体とする. 次の問いに答えよ. 
\begin{enumerate}
  \setlength{\parskip}{0cm}
  \setlength{\itemsep}{2pt} 
 \item $M$が向きづけ可能であるとし, $M$の座標近傍系$\{ (U_{\lambda}, x_{1}^{\lambda}, \ldots, x_{m}^{\lambda})\}_{\lambda \in \Lambda}$を同じ向きになるものとする. $\{U_{\lambda}\}_{\lambda \in \Lambda}$に従属する1の分割を$\{ \rho_{j} \}_{j \in \N}$とするとき
 $$
 \omega = \sum_{j \in\N} \rho_j dx_{1}^{\lambda} \wedge \cdots \wedge x_{m}^{\lambda}
 $$
 はどの点でも0にならない$m$次微分形式であることを示せ. 
 \item 逆にどの点でも0にならない$m$次微分形式$\omega$が存在するならば, $M$が向きづけ可能であることを示せ.\footnote{以上より$M$が向きづけ可能であるための必要十分条件は, どの点でも0にならない$m$次微分形式$\omega$が存在することである. そして向きづけ可能とは$\wedge^{m}T^{*}M$が自明になることと同値である.}
\item $S^2$は向きづけ可能であることを示せ. (ヒント: $S^2$上にどの点でも0にならない$2$次微分形式を構成する. そのような2次微分形式は度々出ている. )
 \end{enumerate}


%多様体$M$が向きづけ可能であるための必要十分条件は, $\wedge^{m} T^{*}M$
%\end{enumerate}

%\item $\C\mathbb{P}^2$は向きづけ可能であることを示せ.\footnote{より一般に複素多様体は向きづけ可能であることがわかる.}


\item $S^n$は向きづけ可能であることを示せ. ただしこの問題は\ref{regular_orin}が解答される前に答えること. 

\item $\C\mathbb{P}^2$は向きづけ可能であることを示せ. \footnote{実はより一般に$\C \mathbb{P}^n$などの複素多様体は向きづけ可能である.}

\item $^*$\label{regular_orin}$f : \R^{n+1} \rightarrow \R$となる$C^{\infty}$級写像で$0 \in \R$が$f$の正則値であるとする. このとき$f^{-1}(0)$は向き付け可能な$n$次元の$C^{\infty}$級多様体であることを示せ. 

\item $^{*}$ $\R \mathbb{P}^n$は$n$が奇数なら向きづけ可能であることを示せ. (ヒント: $p : S^n \to S^n$を$p(\bm{x})=-\bm{x}$とする. $S^n$上のある$n$次微分形式と$p$を用いて, $\R \mathbb{P}^n$上の微分形式でどの点でも0にならないものを作る.)

\item $^{*}$多様体$M$についてその接ベクトル束$TM$は常に向きづけ可能であることを示せ. 

\item $^{*}$ %メビウスの帯を定義し, 向きづけ不可能であることをしめせ. 
$(-1,1) \times \R$に同値関係$\sim$を
	$$
	(x,y) \sim (z,w)\Leftrightarrow \text{ある整数$m$があって}z = (-1)^m x, w = y+m.
	$$
	と定義する. $X := ((-1,1) \times \R) / \sim$とし\underline{メビウスの帯}という. 商写像$\pi : (-1,1) \times \R \rightarrow X$によって$X$に位相を入れる. 次の問いに答えよ.
	\begin{enumerate}
	 \setlength{\parskip}{0cm}
  \setlength{\itemsep}{2pt} 
  \item $U_1: = \pi( (-1,1) \times (0,1) ) $,  $U_2: = \pi( (-1,1) \times (-\frac{1}{2},\frac{1}{2}) )$とおく. 
各$i=1,2$について$\R^2$の開集合$V_i$への同相写像$\varphi_i : U_i \rightarrow V_i$で, $\{ (U_1, \varphi_1), (U_2, \varphi_2)\}$が$X$の座標近傍系になるような$\varphi_1, \varphi_2$を一つ構成せよ. またメビウスの帯は$C^{\infty}$級多様体になることを示せ.
  \item メビウスの帯$X$は向きづけ不可能であることを示せ.
	\end{enumerate}

\item $^{**}$ $\R \mathbb{P}^n$は$n$が偶数なら向きづけ不可能であることを示せ. \footnote{一応\ref{nform}を使えば現時点でも求められる. 他にもホモロジーを使って求めることもできる.(\ref{sec-deRham}章のド・ラームの定理と\ref{fund_alg}を参照のこと.)}


\end{enumerate}


\newpage


\begin{center}
\section{ド・ラーム コホモロジー群}
\label{sec-deRham}
\end{center}
\begin{flushright}
 岩井雅崇(いわいまさたか)
\end{flushright}


    
\begin{tcolorbox}[
    colback = white,
    colframe = green!35!black,
    fonttitle = \bfseries,
    breakable = true]
\begin{dfn}[ド・ラーム コホモロジー (de Rham コホモロジー)]
$M$を多様体とし $k$を0以上の整数とし, $\omega$を$k$次微分形式とする. 
\begin{itemize}
 \setlength{\parskip}{0cm}
  \setlength{\itemsep}{2pt}
\item  $d\omega=0$なる微分形式を\underline{閉形式}という. $k$次微分形式で閉形式であるものからなるベクトル空間を$Z^k(M)$と書く. 
\item ある$k-1$次微分形式$\eta$があって$\omega = d \eta$とかけるとき, $\omega$は\underline{完全形式}と呼ばれる. $k$次微分形式で完全形式であるものからなるベクトル空間を$B^k(M)$と書く. このとき$B^k(M) \subset Z^k (M)$である. つまり完全形式は閉形式である. 
\item $H^{k}_{DR}(M) := Z^k (M)/B^k(M)$とし, \underline{$U$の$k$次ド・ラーム コホモロジー}という
 \end{itemize}
\end{dfn}
\end{tcolorbox}  


\begin{rem}
$d \circ d =0$なので完全形式ならば閉形式である.  ド・ラーム コホモロジー群は閉形式と完全形式のずれを記述している群である.
\end{rem}


\begin{tcolorbox}[
    colback = white,
    colframe = green!35!black,
    fonttitle = \bfseries,
    breakable = true]
\begin{dfn}[完全系列]
$A,B,C$をベクトル空間とし
$$
A \stackrel{f}{\to} B \stackrel{g}{\to} C
$$
となるベクトル束の準同型の系列(sequence)を考える. 
この系列が完全(exact)であるとは$\Ker g = \ima f$となることとする. 
このとき
$$
0 \to A \stackrel{f}{\to} B \stackrel{g}{\to} C \to 0
$$
とかき短完全列(short exact sequence)と呼ばれる. 

また系列
$$
A_0 \stackrel{f_0}{\to} A_1 \stackrel{f_1}{\to} A_2 \stackrel{f_2}{\to}  \cdots \stackrel{f_{n-1}}{\to} A_{n}
$$
が完全(exact)であるとは, $\Ker f_{i}= \ima f_{i-1}$が$i=1, \ldots, n-1$で成り立つこととする. 
\end{dfn}
\end{tcolorbox}  


\begin{tcolorbox}[
    colback = white,
    colframe = green!35!black,
    fonttitle = \bfseries,
    breakable = true]
\begin{thm}[マイヤー・ヴィートリス系列 (Mayer-Vietoris sequence)]
$M$を多様体とし$U, V$を$M$の開被覆とする . 
このとき
\[
\cdots \longrightarrow H^k(M) \longrightarrow H^k(U) \oplus H^k(V) \longrightarrow H^k(U \cap V) \longrightarrow H^{k+1}(M) \longrightarrow \cdots
\]
は完全である. 
\end{thm}
\end{tcolorbox} 

\begin{rema}
トポロジーでならうホモロジーのマイヤー・ヴィートリス系列とは向きが逆になっていることに注意!
\end{rema}


\begin{tcolorbox}[
    colback = white,
    colframe = green!35!black,
    fonttitle = \bfseries,
    breakable = true]
\begin{dfn}[ホモトピック, ホモトピー同値]
$M,N$を多様体とする. 
\begin{itemize}
 \setlength{\parskip}{0cm}
  \setlength{\itemsep}{2pt}
  \item $C^\infty$級写像$f, g: M \to N$がホモトピック(homotopic)であるとは
ある$C^\infty$写像$F : M \times \R \to N$があって
$F(x,0) = f$かつ$F(x,1)=g$を満たすこと. このとき$f \sim g$とかく. 
\item $C^\infty$級写像$f : M \to N$がホモトピー同値(homotopy equivalence)であるとは, ある$C^\infty$級写像$g : N \to M$があって$g \circ f \sim id_{M}$かつ$f \circ g \sim id_{N}$となること. このとき$M$は$N$とホモトピー同値であるという. 
\item $M$が可縮(contractible)であるとは, $M$が1点とホモトピー同値であることとする. 
\end{itemize}
\end{dfn}
\end{tcolorbox}  

\begin{tcolorbox}[
    colback = white,
    colframe = green!35!black,
    fonttitle = \bfseries,
    breakable = true]
\begin{thm}[Tu Theorem 27.10]
$M,N$を多様体とする. $C^\infty$級写像$f, g: M \to N$がホモトピックならば, $k$次ド・ラーム コホモロジーの間の写像$f^{*}$と$g^{*}$は同じ写像である. 
\end{thm}
\end{tcolorbox} 

\begin{tcolorbox}[
    colback = white,
    colframe = green!35!black,
    fonttitle = \bfseries,
    breakable = true]
\begin{cor}[Tu Corollary 27.11]
$M,N$を多様体とする. $C^\infty$級写像$f : M \to N$がホモトピー同値ならば, 
$$
f^{*} : H^{k}_{DR}(M) \to H^{k}_{DR}(N) 
$$
は同型写像である. 
\end{cor}
\end{tcolorbox} 

\begin{tcolorbox}[
    colback = white,
    colframe = green!35!black,
    fonttitle = \bfseries,
    breakable = true]
\begin{cor}[Tu Corollary 27.13 ポアンカレの補題(Poincare lemma)]
$M$が可縮ならば, 1以上の整数$k$について$H^{k}_{DR}(M) =0$. 
特に1以上の整数$k$について$H^{k}_{DR}(\R^m) =0$.
\end{cor}
\end{tcolorbox} 

他に「トゥー 多様体」にはないが有用な定理を述べておく. 以下の内容は「坪井俊 著 幾何学3 微分形式」を参考にした.
 下の定理よりホモロジー群を求めればド・ラーム コホモロジーは求められてしまう.
\begin{tcolorbox}[
    colback = white,
    colframe = green!35!black,
    fonttitle = \bfseries,
    breakable = true]
\begin{thm}[坪井 定理3.3.7 ド・ラーム の定理]
$M$を多様体とき
$$ H^{k}_{DR}(M) \to {\rm Hom} (H_{k}(M,\Z), \R)
$$
はベクトル空間の同型写像である. 
ここで$H_{k}(M,\Z)$は$M$のホモロジー群である. 
\end{thm}
\end{tcolorbox} 


%問題の上に$^{\bullet}$がついている問題は\underline{解けてほしい}問題である. 問題の上に$^{*}$がついている問題は\underline{面白いかちょっと難しい}問題である. 

%以下断りがなければ$M,N$は$C^{\infty}$級多様体とし, $m = \dim M$とする.$\R^n$をユークリッド空間とし, $S^n \subset \R^{n+1}$を半径1の$n$次元球面とする.

以下断りがなければ$M$には境界がないものとする. (つまり$\partial M = \varnothing$を仮定する. )
\vspace{11pt}
\begin{enumerate}[label=\textbf{問}\ref*{sec-deRham}.\arabic*]

\item  $^\bullet$ 連結な多様体$M$について0次ド・ラーム コホモロジー群$H^{0}_{DR}(M)$を求めよ. 

%\item $\bullet$ $m$次元コンパクト連結多様体$M$について, $M$が向き付け可能ならば$H^{m}_{DR}(M) \neq 0$であることを示せ. 

\item $^\bullet$ $H^{1}_{DR}(S^1) \neq 0$であることを示せ. ただしこの問題は\ref{circle_cohomology}が解答される前に答えること. 

\item $^\bullet$ $\dim H^{1}_{DR}(\R^2 \setminus \{(0,0) , (1,0)\} ) \geqq 2$であることを示せ. ただしこの問題は\ref{circle_2_cohomology}が解答される前に答えること. 

\item $^\bullet$ $\R$ベクトル空間の完全列
  $$
0 \to A_1 \stackrel{f_1}{\to} A_2 \stackrel{f_2}{\to} A_3\stackrel{f_3}{\to} A_4  \to 0
  $$
  を考える. 次の問いに答えよ. 
\begin{enumerate}
 \setlength{\parskip}{0cm}
  \setlength{\itemsep}{2pt}
  \item 
  %$\R$ベクトル空間の完全列$0 \to A_1 \stackrel{f_1}{\to} A_2 \stackrel{f_2}{\to} A_3 \to 0$について
  $\dim A_1 =2, \dim A_2=5, \dim A_4=0$のとき$\dim A_3$を求めよ. 
    \item %$\R$ベクトル空間の完全列$$0 \to A_1 \stackrel{f_1}{\to} A_2 \stackrel{f_2}{\to} A_3\stackrel{f_3}{\to} A_4  \to 0$$について
  $\dim A_1 =2, \dim A_2=5, \dim A_3=4$のとき$\dim A_4$を求めよ. 
  \end{enumerate}
  \newpage 
 \item $^\bullet$
 %次の問いに答えよ.
%\begin{enumerate}
 %\setlength{\parskip}{0cm}
 % \setlength{\itemsep}{2pt}
%  \item $\R$ベクトル空間の系列
 % $$
%0 \to A_1 \stackrel{f_1}{\to} A_2\to 0
 % $$
%が完全であることは$f_1$が同型であることと同値であることを示せ. 
   % \item 
    $\R$ベクトル空間の系列
  $$
0 \to \R \stackrel{f_1}{\to} \R^2 \stackrel{f_2}{\to} \R \stackrel{f_3}{\to} A_4 \stackrel{f_4}{\to} A_5 \to 0
  $$
が完全であるとき, $f_4$は同型であることを示せ. 
%  \end{enumerate}
  
\item \label{circle_cohomology}$^\bullet$ 次の問いに答えよ.
%$S^1$のド・ラーム コホモロジーをマイヤー・ヴィートリス系列を使って求める. 次の問いに答えよ.
\begin{enumerate}
 \setlength{\parskip}{0cm}
  \setlength{\itemsep}{2pt}
  \item $U = S^1 \setminus \{(1,0)\}$, $V = S^1 \setminus \{(-1,0)\}$とする. $H^{k}_{DR}(U)$と$ H^{k}_{DR}(V)$ ($k=0,1,2,\ldots$)をそれぞれ求めよ.
  \item $U \cap V$は$\R \setminus 0$と微分同相であることを示し, $H^{k}_{DR}(U \cap V)$ ($k=0,1,2,\ldots$)を求めよ.
  \item $S^1$のド・ラーム コホモロジー群$H^{k}_{DR}(S^1)$($k=0,1,2,\ldots$)を求めよ.
\end{enumerate}

\item $^\bullet$ $S^n$のド・ラーム コホモロジー群$H^{k}_{DR}(S^n)$($k=0,1,2,\ldots$)を求めよ.

\item  $^\bullet$ $\R^m$上の開集合$U$が星型であるならば可縮であることを示せ. ここで$U$が星型であるとはある点$p \in U$があって, 任意の点$x$と任意の$t \in [0,1]$について$(1-t)p + t x \in U$が成り立つこととする. 

\item  $^\bullet$ $\R^2 \setminus \{ (0,0) \}$のド・ラーム コホモロジー群$H^{k}_{DR}(\R^2 \setminus \{ (0,0) \})$($k=0,1,2,\ldots$)を求めよ. (ヒント: $\varphi : \R^2 \setminus \{(0,0)\}\to S^1$で$\varphi(\bm{x}) = \frac{\bm{x}}{|\bm{x}|}$を考え,  $\R^2 \setminus \{(0,0)\}$と$S^1$がホモトピー同値であることを示す. )




\item \label{circle_2_cohomology} $\R^2 \setminus \{(0,0) , (1,0)\} $のド・ラーム コホモロジー群$H^{k}_{DR}(\R^2 \setminus \{ (0,0), (1,0) \})$($k=0,1,2,\ldots$)を求めよ.

\item $Z = \{ x=y=0\} \subset \R^3$とする. $\R^3 \setminus Z$のド・ラーム コホモロジー群$H^{k}_{DR}( \R^3 \setminus Z)$($k=0,1,2,\ldots$)を求めよ.

\item $2$次元トーラス$T^2 = \R^2/ \Z^2$のド・ラーム コホモロジー群$H^{k}_{DR}(T^2)$($k=0,1,2,\ldots$)を求めよ.

\item \label{real_plane_coho} $\R \mathbb{P}^2$のド・ラーム コホモロジー群$H^{k}_{DR}(\R \mathbb{P}^2)$($k=0,1,2,\ldots$)を次の方法で求めよ.
\begin{enumerate}
 \setlength{\parskip}{0cm}
  \setlength{\itemsep}{2pt}
  \item $\pi : S^2 \to \R \mathbb{P}^2$を自然な射影とする. $\pi(\{ z=0\} \cap S^2)$は$\R \mathbb{P}^1$とみなせることを示せ.以降$\pi(\{ z=0\} \cap S^2)$を$\R \mathbb{P}^1$とかく. 
  \item $U = \R \mathbb{P}^2 \setminus \R \mathbb{P}^1$と$V = \R \mathbb{P}^2 \setminus \pi((0,0,1))$のド・ラーム コホモロジー群を求めよ. 
  \item $U \cap V$は$S^1$とホモトピー同値であることをしめせ.
  \item $H^{k}_{DR}(\R \mathbb{P}^n)$($k=0,1,2,\ldots$)を求めよ. (ヒント: $H^1(U) \oplus H^1(V) \to H^1(U \cap V)$は同型になる. )
 \end{enumerate}

\item $^{**}$ $\R \mathbb{P}^n$のド・ラーム コホモロジー群$H^{k}_{DR}(\R \mathbb{P}^n)$($k=0,1,2,\ldots$)を求めよ. \footnote{おそらくホモロジーを求めてド・ラームの定理を使う方が楽かもしれない.}

\item $^{*}$  $\C \mathbb{P}^n$のド・ラーム コホモロジー群$H^{k}_{DR}(\C\mathbb{P}^n)$($k=0,1,2,\ldots$)を求めよ. (ヒント: $H = \{ z_1=0\}$を考え\ref{real_plane_coho}と同じことをする.)
%\item  $^{*}$ $S^1$のド・ラームコホモロジー群$H^{k}_{DR}(S^1)$($k=0,1,2,\ldots$)を求めよ.
%\footnote{頑張れば今の状況でも求められる.(トゥー 多様体24章を見よ.) 他にも\underline{ド・ラームの定理}「滑らかな多様体$M$について$H^{k}_{DR}(M) \cong {\rm Hom}(H_{k}(M,\Z), \R)$が成り立つ」を認めればホモロジー群から求められる. }

\item $^{*}$ 次の問いに答えよ.
\begin{enumerate}
 \setlength{\parskip}{0cm}
  \setlength{\itemsep}{2pt}
\item $(M , \omega)$をコンパクトシンプレクティック多様体とするとき, $1 \le k \le n$なる自然数について$H^{2k}_{DR}(M) \neq 0$であることを示せ. 
\item $S^n$がシンプレクティック多様体になるような$n$の値を決定せよ. 
\end{enumerate}

\item 次の問いに答えよ.
\begin{enumerate}
 \setlength{\parskip}{0cm}
  \setlength{\itemsep}{2pt}
\item 有限ベクトル空間の完全系列
$$
0 \to A_0 \stackrel{f_0}{\to} A_1 \stackrel{f_1}{\to} A_2 \stackrel{f_2}{\to}  \cdots \stackrel{f_{n-1}}{\to} A_{n} \to 0
$$
について$\sum_{i=0}^{n} (-1)^i \dim A_i =0$であることを示せ.(ヒント: まず$n=2$のときを考える.)
\item 多様体$M$についてオイラー標数を
$$
\chi (M) = \sum_{i=0}^{\dim M} (-1)^i \dim H^{i}_{DR}(M)
$$
として定める. 
$M$の開被覆$U, V$について, $M, U,V, U\cap V$が全て有限次元$k$次ド・ラーム コホモロジーを持つならば, 
$
\chi(M) - (\chi(U) + \chi(V) ) + \chi (U \cap V) =0
$
であることを示せ. 
\item $\chi(S^n)$を求めよ. \footnote{余力があればオイラーの多面体定理との関係を調べよ. }

\end{enumerate}

\item \label{fund_alg} $^{*}$ $n$を1以上の整数とし, $\C$係数の多項式$p(z) = z^n + a_{1}z^{n-1} + \cdots + a_n$で$a_n\neq 0$なるものを考える. 
$F_0, F_1 : \C\mathbb{P}^1 \to  \C\mathbb{P}^1$を次のように定める. 
$$
F_0 (z: 1) = (z^n : 1), \quad F_0 (1: 0) = (1 : 0), \quad F_1 (z: 1) = (p(z) : 1), \quad F_1(1: 0) = (1 : 0).
$$
以下の問いに答えよ. 
\begin{enumerate}
 \setlength{\parskip}{0cm}
  \setlength{\itemsep}{2pt}
  \item $F_0, F_1$は$C^{\infty}$級写像であることを示せ.
  \item $F_0$と$F_1$はホモトピックであることを示せ. 
 \item 下の定理から$I : H^{2}_{DR}(\C\mathbb{P}^1) \simeq \R$である.  この対応は$I(\omega) = \int_{\C\mathbb{P}^1} \omega$であることもわかっている. このとき
 $$
F_{0}^{*}:  H^{2}_{DR}(\C\mathbb{P}^1) \to  H^{2}_{DR}(\C\mathbb{P}^1)
 $$ 
 を求めよ. つまり$F_{0}^{*}: \R \to \R$とみなしたとき, これは何倍の写像になっているか? 
 \item $p(z) =0$となる点が$z \in \C$に存在しなければ, $F_1$は$(1:0)$に移す定値写像とホモトピックであることを示せ.
 \item 代数学の基本定理「$\C$係数$n$次多項式$p(z)$は$\C$内に解を持つ」を示せ. 
\end{enumerate}
この問題に限り次の定理を認めて良い. 
\begin{tcolorbox}[
    colback = white,
    colframe = green!35!black,
    fonttitle = \bfseries,
    breakable = true]
\begin{thm}[坪井 定理3.4.11]
$M$を境界を持たない$m$次元コンパクト連結多様体とする. 
$H^{m}_{DR}(M)$が$\R$であることは$M$が向きづけ可能であることと同値である. 
また$H^{m}_{DR}(M)$が$0$であることは$M$が向きづけ不可能であることと同値である. 
\end{thm}

\end{tcolorbox} 



\end{enumerate}









 \end{document}
 
 
 
 
 
 
 
 
 
 
 
 
 
 
 
 
 
 
 
 
 
 

%%%%%%%%%%%%%%%%%%%%%%%%%%%%%%%%%%%%%%%%%%

\begin{comment}


\vspace{11pt}
\begin{wrapfigure}{r}[0pt]{0.2\textwidth}
  \centering
 \includegraphics[height=25mm, width=25mm]{stokes.png}
\end{wrapfigure}

演習の問題は授業ページ(\url{https://masataka123.github.io/2022_winter_generaltopology/})にもあります. 
右のQRコードからを読み込んでも構いません.



\begin{wrapfigure}{r}[0pt]{0.5\textwidth}
%\begin{flushright}
 \includegraphics[height=17mm, width=17mm]{stokes.png}
 %\caption*{}
%\end{flushright}
\end{wrapfigure}
\section{微分形式の計算規則}
座標近傍$(U, x_1, \ldots, x_m)$とする.\footnote{あるいは$M$を$\R^m$の開集合と考えても良い.}

    \begin{tcolorbox}[
    colback = white,
    colframe = green!35!black,
    fonttitle = \bfseries,
    breakable = true]
\begin{dfn}
$V$を$m$次元の$\R$ベクトル空間とする.
\begin{itemize}
\item $V$上の$k$次多重線型形式とは$\omega : V \times \cdots \times V \rightarrow \R$となる写像で$\omega(v_1, \ldots, v_k)$が各々$X_i$について線型であることとする. $V$上の$k$次多重線型形式なす$m^k$次元のベクトル空間を$\otimes^{k} V^{*}$と表す.
%\item $\eta_1, \ldots, \eta_k \in V^{*}$について$$\begin{matrix} \eta_1\otimes \cdots \otimes \eta_k  : &V \times \cdots \times V & \rightarrow &\R\\&(v_1, \ldots, v_k)& \mapsto& \eta_1(v_1)\cdots \eta_{k}(v_k)\end{matrix}$$と定義し, $\eta_1\otimes \cdots \otimes \eta_k$を$\eta_1, \ldots, \eta_k$のテンソル積という.
  %\item 特に$\{ e_1, \ldots, e_m\}$を$V$の基底とし, $\{ \omega_1, \ldots, \omega_m\}$は$V^{*}$を $\{ e_1, \ldots, e_m\}$の双対基底とするとき, $\{ \omega_{i_1}\otimes \cdots \otimes  \omega_{i_k}\}_{i_1, \ldots, i_k=1, \ldots,m}$が$\otimes^{k} V^{*}$の基底となる.
  \item $\omega \in \otimes^{k} V^{*}$が$k$次対称形式であるとは, 任意の$k$次の置換$\sigma$と, 任意の$\omega(v_1, \ldots, v_k) \in V^{k}$について$\omega(v_{\sigma(1)}, \ldots, v_{\sigma(k)})=\omega(v_1, \ldots, v_k)$となること
  \item $\omega \in \otimes^{k} V^{*}$が$k$次交代形式であるとは, 任意の$k$次の置換$\sigma$と, 任意の$\omega(v_1, \ldots, v_k) \in V^{k}$について$\omega(v_{\sigma(1)}, \ldots, v_{\sigma(k)})=\sgn(\sigma)\omega(v_1, \ldots, v_k)$となること. $V$の$k$次交代形式の${}_m C_{k}$次元のベクトル空間を$\wedge^{k} V^{*}$で表す. 
\end{itemize}
    \end{dfn}
    \end{tcolorbox}
\begin{exa}
 $\eta_1, \ldots, \eta_k \in V^{*}$について
     $$
      \begin{matrix}
     \eta_1\otimes \cdots \otimes \eta_k  : &V \times \cdots \times V & \rightarrow &\R &
     \eta_1\wedge \cdots \wedge \eta_k  : &V \times \cdots \times V & \rightarrow &\R \\
      &(v_1, \ldots, v_k)& \mapsto& \eta_1(v_1)\cdots \eta_{k}(v_k) & &(v_1, \ldots, v_k)& \mapsto& \det((\eta_i(v_j))_{1 \le i,j \le k})
       \end{matrix}
      $$
      と定義する. $\eta_1\otimes \cdots \otimes \eta_k \in \otimes^{k} V^{*}$であり$\eta_1, \ldots, \eta_k$のテンソル積と呼ばれる. 

$\{ e_1, \ldots, e_m\}$を$V$の基底とし, $\{ \omega_1, \ldots, \omega_m\}$は$V^{*}$を $\{ e_1, \ldots, e_m\}$の双対基底とするとき, $\{ \omega_{i_1}\otimes \cdots \otimes  \omega_{i_k}\}_{i_1, \ldots, i_k=1, \ldots,m}$が$\otimes^{k} V^{*}$の基底となる.また$\{ \omega_{i_1}\wedge \cdots \wedge \omega_{i_k}\}_{1 \le i_1< \cdots < i_k<m}$が$\wedge^{k} V^{*}$の基底となる.
\end{exa}

\item 写像$f : \C^n \rightarrow \C^{n}$を
     $$
      \begin{matrix}
     f : &\C^n & \rightarrow &\C^n\\
      &z=(z_1, \ldots, z_n) & \mapsto& (f_1(z), \ldots, f_n(z))
       \end{matrix}
      $$
%写像$f : \C^n \rightarrow C^{n}$, $f(z_1, \ldots, z_n) = (f_1(z), \ldots, f_n(z))$とする. を各変数について正則であるとする.
とし, 各$f_i(z)$は各変数$z_1, \ldots, z_n$について正則であるとする. 次の問いに答えよ.
\begin{enumerate}
 \setlength{\parskip}{0cm}
  \setlength{\itemsep}{2pt} 
\item $z_i = x_i + \sqrt{-1}y_i$によって$\C^n$に座標$(x_1, y_1, \ldots, x_n, y_n)$を入れる. $f$のヤコビ行列の行列式は常に0以上であることをしめせ.
\item $\C\mathbb{P}^n$は向きづけ可能であることを示せ.\footnote{より一般に複素多様体は向きづけ可能であることがわかる.}
\end{enumerate}


\item $^{*}$ $X$をベクトル場とし, $\omega$を$k$次微分形式とする.
$$
(L_{X}\omega) (X_1,\ldots, X_k):=X(\omega(X_1, \ldots, X_k)) - \sum_{i=1}^{k}\omega(X_1, \ldots, [X,X_i], \ldots, X_k)
$$
と定義する. \underline{$L_{X}\omega$を$\omega$の$X$によるLie微分}という. 次の問いに答えよ. \footnote{難しければ$k=1$など低い次数の微分形式対して示して良い. }
\begin{enumerate}
 \setlength{\parskip}{0cm}
  \setlength{\itemsep}{2pt} 
\item $L_X \omega$は$k$次微分形式であることを示せ. 
\item $\{ \varphi_{t} \}_{t \in \R}$を$X$の1パラメーター変換群とするとき, $L_{X} \omega = \lim_{t \rightarrow 0}\frac{\varphi^{*}_{t} \omega - \omega}{t}$を示せ. 
\item $L_{X}L_{Y} - L_{Y}L_{X}= L_{[X,Y]}$を示せ.
\item $L_{X}(\omega \wedge \eta )=L_{X}(\omega) \wedge \eta  + \omega \wedge L_{X}( \eta )$と$d L_{X} = L_{X}  d$をそれぞれ示せ.
\end{enumerate}

\item $^{*}$ $X$をベクトル場とし, $\omega$を$k$次微分形式とする.
$$
(i_{X}\omega) (X_1,\ldots, X_{k-1}):=\omega(X, X_1, \ldots, X_{k-1}) 
$$
と定義する. \underline{$i_{X}\omega$を$\omega$と$X$の内部積}という. 次の問いに答えよ
\begin{enumerate}
 \setlength{\parskip}{0cm}
  \setlength{\itemsep}{2pt} 
\item $i_X \omega$は$k-1$次微分形式であることを示せ. 
%\item $\{ \varphi_{t} \}_{t \in \R}$を$X$の1パラメーター変換群とするとき, $L_{X} \omega = \frac{\varphi^{*}_{t} \omega - \omega}{t}$であることを示せ. 
%\item $L_{X}L_{Y} - L_{Y}L_{X}= L_{[X,Y]}$を示せ.
\item $\omega$を$k$次微分形式とするとき, $i_{X}(\omega \wedge \eta )=i_{X}(\omega) \wedge \eta  +(-1)^k \omega \wedge i_{X}( \eta )$を示せ. 
\item $i_{[X,Y]} = L_{X} i_{Y} - i_{Y} L_X$を示せ.
\item Cartanの公式 $L_X = i_{X}  d + d  i_{X} $を示せ. 
\end{enumerate}

\item $^{*}$ $\omega$を$\R^n$上の1次微分形式とし, $S_{\omega}$を$\R^n$のベクトル場$X$で$i_{X}\omega =0$となるものの集合とする.  $d \omega \wedge \omega =0$ならば任意の$X,Y \in S_{\omega}$について$[X,Y] \in S_{\omega}$であることを示せ.
%次が同値であることを示せ.%\footnote{微分形式を用いた葉層の特徴づけである. }
%\begin{enumerate}
% \setlength{\parskip}{0cm}
 % \setlength{\itemsep}{2pt} 
%\item 任意の$X,Y \in S_{\omega}$について$[X,Y] \in S_{\omega}$である.
%\item $d \omega \wedge \omega =0$.
%\end{enumerate}


\item $^{*}$ $M$を向きづけ可能なコンパクト多様体とする. 
$g$が$M$上のリーマン計量であるとは, 任意の点$p\in M$について$g_p$が$T_{p}M$の正定値対称2次形式であるものとする. (多様体の基礎19参照). $\{ (U_{\lambda}, x_{1}^{\lambda}, \ldots, x_{m}^{\lambda})\}_{\lambda \in \Lambda}$を互いに同じ向きになる座標近傍系とする. 
$g_{ij}^{\lambda} =g \left(\pdrv{}{x_{i}^{\lambda}}, \pdrv{}{x_{j}^{\lambda}}\right)$とし, 
$$
\omega_{\lambda} := \sqrt{\left|\det (g_{ij}^{\lambda}) \right|} dx_{1}^{\lambda} \wedge \cdots \wedge d x_{m}^{\lambda}
$$
とおくと$M$上の$m$次微分形式$\omega_{g}$で$\omega_{g}|_{U_{\lambda}} = \omega_{\lambda} $となるものが存在することを示せ. この$\omega_g$は\underline{リーマン計量の体積要素}と呼ばれる.


\item $\R^2$に標準的なリーマン計量$g$, つまり$g_{ij} := g\left(\pdrv{}{x_{i}^{\lambda}}, \pdrv{}{x_{j}^{\lambda}}\right) = \delta_{ij}$となる計量$g$を入れる. \footnote{$\delta_{ij}$はクロネッカーのデルタである.}
$f : [] \to \R$を$$



%\item 半径1の開円盤を$B^2$とし$B^2$に$\R^2$から誘導されるリーマン計量$g_{B^2}$を入れる時, $\int_{B^2} \omega_{g_{B^2}}$を求めよ.
\item $S^1$に$\R^2$から誘導されるリーマン計量$g_{S^1}$を入れる. $\int_{S^1} \omega_{g_{S^1}}$を求めよ.

\item $S^2$に$\R^3$から誘導されるリーマン計量$g_{S^2}$を入れる. $\int_{S^2} \omega_{g_{S^2}}$を求めよ.

%\item 半径1の開球を$B^{3}$とし, $B^{3}$に$\R^3$から誘導されるリーマン計量$g_{B^3}$を入れる. $\int_{B^3} \omega_{g_{B^3}}$を求めよ.


\item $^{**}$ (学部一年の積分の授業で習ったと思われる)曲線の長さや曲面の表面積を求める公式を上のリーマン計量の体積要素を用いて導出せよ. \footnote{学部1年で線分の長さや表面積の公式習ったと思うが, その公式の証明はされていなかったと思う. 実は表面積や線分の長さの公式はリーマン計量の体積要素からわかるものであり, とどのつまり学部3年にしてようやく表面積や曲線の長さが定義できたのである. }
%$$S_{\omega} := \{ X \in \chi(\R^n)  | i_{X} \}$$
%\item ベクトル解析
%\item カルタンの公式
%\item 葉層の問題
%\item 電磁気の法則

%\item $^{**}$ベクトル解析におけるガウスの発散定理を調べ, それが(多様体の)ストークスの定理から導かれることを示せ.\footnote{面積分をうまく定義する必要がある. 本当はもっと演習問題ぽく出したかったがどうもリーマン計量が出てくるためうまく問題が作れなかった...} 



\end{comment}
 
 
%%%%%%%%%%%%%%%%%%%%%%%%%%%%%%%%%%%%%%%%%%%%%%%%
\begin{comment}
%\item 次を示せ.
	%\begin{enumerate}
	\item $^{*}$ Give an example of a topological space that is connected but not path-connected. %連結だが弧状連結でない位相空間の例をあげよ.
	\item $^{*}$ Show that any connected manifold is path-connected. %連結な多様体は弧状連結であることをしめせ. 
	%\end{enumerate}
\item 次の問いに答えよ.
	\begin{enumerate}
	\item 実数の集合$\R$について, 同値関係$\sim_{1}$を
	$$
	x \sim_{1} y \Leftrightarrow x - y \in \Z
	$$
	とし$\R / \sim_{1}$に商位相を入れる. このとき$\R / \sim_{1}$は多様体になることを示せ.
	\item 実数の集合$\R$について, 同値関係$\sim_{2}$を
	$$
	x \sim_{2} y \Leftrightarrow x - y \in \Q
	$$
	とし$\R / \sim_{2}$に商位相を入れる.  このとき$\R / \sim_{2}$は多様体とならないことを示せ.
	\end{enumerate}
	

\item $f : \C \rightarrow \C$を$f(z) = z(z+1)$とする. 次の問いに答えよ.
	\begin{enumerate}
	\item $z = x + \sqrt{-1} y$によって$\C$に座標$(x,y)$を入れ$f$を座標表示せよ.
	\item $z \in \C$においてヤコビ行列を求めよ.
	\item $(df)_{p} : T_{z}\C \rightarrow T_{z}\C$が同型でない$z$を全て求めよ.
	%\item $i :\C \rightarrow \C\mathbb{P}^{1}$を$i(z) = (z:1)$とすることにより, $\C$を$\C\mathbb{P}^{1}$の開部分多様体と見なす.  ある$F : \C\mathbb{P}^{1} \rightarrow \C\mathbb{P}^1$となる$C^{\infty}$級写像で$F|_{\C} = f$となるものがあることを示せ. 
	\end{enumerate}

\item %$S^2 = \{ (x_1, x_2, x_3)| x_{1}^{2} + x_{2}^{2} + x_{3}^{2}= 1 \}$とおく. 
$\C\mathbb{P}^{1}$と$S^2$は$C^{\infty}$級微分同相であることをしめせ. 

\item (多様体の基礎 11章)
	\begin{enumerate}
	\item  
	$$
\begin{array}{ccccc}
\pi: &S^{2n+1}& \rightarrow & \C\mathbb{P}^{n} & \\
&(x_1, x_2, \ldots, x_{2n+1}, x_{2n+2}) & \longmapsto &(x_1 + \sqrt{-1}x_2, x_3 + \sqrt{-1}x_4,\ldots, x_{2n+1}+ \sqrt{-1}x_{2n+2})&
\end{array}
$$
とおく. この写像が全射$C^{\infty}$級写像であることを示せ.
\item $\C\mathbb{P}^{n} $はコンパクトであることを示せ. 
	\item 任意の$z \in \C\mathbb{P}^{n}$について$f^{-1}(z)$は$S^{1}$と位相同相であることを示せ.
	\end{enumerate}

\item $^{*}$ $n$を2以上の整数とする. $H = \{ (z_{1}:z_{2}: \ldots : z_{n+1}) \in  \C\mathbb{P}^{n} |  z_{1} + \cdots+ z_{n+1} =0\}$が$C^{\infty}$級多様体であることを示し, その次元を求めよ.

\item $^{**}$ $n$を2以上の整数とする. $Q= \{ (z_{1}:z_{2}: \ldots : z_{n+1}) \in \C\mathbb{P}^{n} |  z_{1}^{2} + \cdots +z_{n+1}^{2} =0\}$が$C^{\infty}$級多様体であることを示し, その次元を求めよ.



\item  (多様体の基礎 15章) $O(n, \R) = \{ A \in M(n,\R) | {}^{t}AA =E\}$が$C^{\infty}$級多様体であることを示し, その次元を求めよ. 
	\item  $SO(n, \R) = \{ A \in M(n,\R) | \det A =1, {}^{t}AA =E\}$が$C^{\infty}$級多様体であることを示し, その次元を求めよ. 
	\item $SO(2, \R) $が$S^1$と$C^{\infty}$級微分同相であることを示せ. 
	
\item  $^{*}$ 次の問いに答えよ
	\begin{enumerate}
	\item $GL(n, \R) $は弧状連結ではないことを示せ.
	\item $GL(n, \R)_{+}=\{ A \in M(n,\R) | \det A > 0\}$は弧状連結であること示せ.
	\end{enumerate}

\item $^{*}$ $SO(n, \R) $は弧状連結であることを示せ.

\item $^{*}$ 
$$
M = \{ (x,y,z,w) \in\R^4  | 2x^2 + 2 = 2 z^2 + w^2, 3x^2 + y^2 = z^2 + w^2\}
$$
とおく. 次の問いに答えよ.
	\begin{enumerate}
	\item $M$は$\R^4$の部分多様体であることを示し, その次元を求めよ
	\item $F : M \rightarrow \R^2$を$F(x,y,z,w) = (x^2, y^2)$とする. $p=(X,Y) \in \R^2$について$F^{-1}(p)$の元の個数を求めよ.
	\item $M$はコンパクトかどうか判定せよ
	\end{enumerate}
	
\item $^{**}$ $G_{4,2}$は$\{ (z_0:z_1:z_2:z_3:z_4:z_5) \in  \C\mathbb{P}^{5}| z_0z_5 - z_1z_4 + z_2z_3 =0\}$と$C^{\infty}$級同相であることを示せ. (プリュッカー埋め込みと呼ばれる).



\item$^{*}$ 授業や演習などこれまで出てきた多様体の例以外で面白い多様体の例をあげよ. ただし以下の点に注意すること.
	\begin{enumerate}
	\item この問題は教官とTAが「面白い」と思わない場合, 正答とならない. (例えば$\R^4$の開集合やトーラス・メビウスの帯・クラインの壺などはよく見るので正答とはならない.)
	\item この問題は複数人が解答して良い.
	\item この問題の解答権は2022年10月中とする. 11月以後はこの問題に答えることはできない. 
	\end{enumerate}


\item $i : S^{m} \rightarrow \R^{m+1} $を包含写像とする. 次の問いに答えよ.
\begin{enumerate}
	\item 任意の点$a \in S^{m}$について, 微分写像$(di)_{a} : T_{a}S^{m} \rightarrow T_{a}\R^{m+1}$は単射であることをしめせ.
	\item $a \in S^{m}$を$S^{m}$の点とする. $(di)_{a} $が単射であることと$T_{a}\R^{m+1} \cong \R^{m+1}$により$T_{a}S^{m}  \subset \R^{m+1} $とみなす. 
	このとき
	$$
	T_{a}S^{m} = \{ v \in \R^{m+1} | <a,v> = 0\}
	$$
	となることを示せ. ここで$<\bullet, \bullet>$は$ \R^{m+1}$上のユークリッド内積とする. 
\end{enumerate}
 
 \item$^{**}$上の記法において$g : S^2 \rightarrow \R^4$を$g(x,y,z)=(yz,zx,xy, x^2+2y^2 + 3z^2)$とする. $g$と$\pi$を使って自然に$\tilde{g}: \R\mathbb{P}^{2} \rightarrow \R^4$が定義でき, $\tilde{g}$は埋め込みであることを示せ.



\end{comment}
%%%%%%%%%%%%%%%%%%%%%%%%%%%%%%%%%%%

%%%%%%%%%%%%%%%%%%%%%%%%%%%%%%%%%%%
\begin{comment}
\begin{center}
\section{ユ}
\label{sec-euc}
\end{center}
\begin{flushright}
 岩井雅崇(いわいまさたか)
\end{flushright}

以下断りがなければ$M,N$は$C^{\infty}$級多様体とし, $m = \dim M$とする.

\vspace{11pt}
\hspace{-22pt}{\large $\bullet$ベクトル場の問題}
\begin{enumerate}[label=\textbf{問}\ref*{sec-euc}.\arabic*]

\item 次の問いに答えよ.
\begin{enumerate}
\item $X$を$C^{\infty}$級ベクトル場とし$f$を$M$上の$C^{\infty}$関数とする. $Xf$と$fX$の厳密な定義とその違いを述べよ.
\item $\R^{2}$上でのベクトル場のかっこ積$[- y \pdrv{}{x} + x \pdrv{}{y}, \pdrv{}{x}]$を計算せよ.
\end{enumerate}



\item 
$a,b \in \R$と$X,Y,Z \in \mathscr{X}(M)$について, 次が成り立つことを示せ.\footnote{$(\mathscr{X}(M), [ , ])$がリー代数の構造をもつ}
\begin{enumerate}
\item (双線型性) $[aX+bY, Z]=a[X,Z] + b[Y,Z], [Z, aX+bY]=a[Z,X] + b[Z,Y] $.
\item (交代性) $ [Y,X]=-[X,Y]$.
\item (ヤコビ恒等式) $[[X,Y],Z] + [[Y,Z],X] +[[Z,X],Y] = 0$.
\end{enumerate}




\item \label{sn_no_vanishing} $^{*}$ Let $(x_1, y_1, \ldots, x_{n}, y_{n})$ be the standard coordinates on $\R^{2n}$. The unit sphere $S^{2n-1}$ in $\R^{2n}$ is defined by the equation $\sum_{i=1}^{n}x_{i}^{2} + y_{i}^{2} =1$.
Show that
$$
X = \sum_{i=1}^{n} - y_i \pdrv{}{x_i} + x_i \pdrv{}{y_i}
$$ 
is a no where-vanishing $C^{\infty}$ vector field on $S^{2n-1}$.


%\item 以下の問いに答えよ.ただし多様体の基礎 命題13.11は使って良い. \footnote{多様体の基礎を読んでいて以下の部分が欠落していると思った}
	%\begin{enumerate}
	%\item 「ベクトル場$X$が$C^{\infty}$級である」ことは「任意の$C^{\inftyl}$級関数$f$について$Xf$が$C^{\infty}$級であること」と同値である
	%\item $C^{\infty}$級関数
	%\item $\varphi_{*}[X,Y] = $
	%\end{enumerate}




\item \label{tm_vector} $^{*}$ 引き続き接ベクトル束$TM$に関する次の問いに答えよ.
\begin{enumerate}
\item $\pi : TM \rightarrow M$は全射$C^{\infty}$級写像であることを示せ. 
\item 「$C^{\infty}$ベクトル場$X$」は「$C^{\infty}$級写像$\chi : M \rightarrow TM$で$\pi \circ \chi = id_{M}$となるもの」と1対1に対応することを示せ.
\item $M$上の$C^{\infty}$ベクトル場$X_1, \ldots, X_{m}$で, 任意の$p \in M$について$(X_1)_{p}, \ldots, (X_{m})_{p}$が$T_{p}M$の基底となるものが存在すると仮定する. このとき$TM $と$M \times \R^{m}$は微分同相であることを示せ.  
\end{enumerate}

\item $TS^1$は$S^1 \times \R$と微分同相であることを示せ. ただしこの問題の解答期限は\ref{sn_no_vanishing}と\ref{tm_vector}が解かれるまでとする. \footnote{\ref{sn_no_vanishing}と\ref{tm_vector}(c)から$TS^1 $と$S^{1} \times \R $は微分同相であることがいえる. もし別解があれば発表してもよい.}

\item $^{*}$  $TS^n$は$\{ (z_1, \ldots,z_{n+1}) \in \C^{n+1} | \sum_{i=1}^{n+1} z_{i}^{2} =1\}$と微分同相であることを示せ. \footnote{ヒント. 前回演習の問1.5を用いる.}
%またこれを用いて$TS^1 $と$S^{1} \times \R $は微分同相であることを示せ.\footnote{ヒント. 前回演習の問1.5を用いる.}

%\item 一時独立なベクトル場の問題

% ベクトル束が張り合うとは?
\item $^{*}$  $i : \C \rightarrow \C\mathbb{P}^{1}$を$i(z) = (z:1)$とすることにより, $\C$を$\C\mathbb{P}^{1}$の開部分多様体と見なす.  
$\C $上のベクトル場$X = x \pdrv{}{x} + y \pdrv{}{y}$と定める (ただし$z = x + \sqrt{-1} y$として$(x,y)$を$\C$の座標を考えている).
このとき$X$は$\C \mathbb{P}^{1}$上の$C^{\infty}$級ベクトル場$\tilde{X}$に拡張されることを示せ. また$\tilde{X}_{p}=0$となる$p \in \C \mathbb{P}^{1}$を全て求めよ. 

\item Let $F : M \rightarrow N$ be a $C^{\infty}$ diffeomorphism of manifolds. 
\begin{enumerate}
\item Prove that if $g$ is a $C^{\infty}$ function and $X$ is a $C^{\infty}$ vector field on $M$, then $F_{*}(gX) = (g \circ F^{-1}) F_{*}X$.
\item Prove that if $X$ and $Y$ are $C^{\infty}$ vector fields on $M$, then $F_{*}[X,Y]=[F_{*}X,F_{*}Y]$.

\end{enumerate}


%\newpage
\vspace{11pt}
\hspace{-33pt}{\large $\bullet$積分曲線の問題}

%\item Let $X$ be the vector field $x^{2}\pdrv{}{x}$ on the real line $\R$. Find the maximal integral curve of $X$ starting at $x=2$.

\item 
\begin{enumerate}
\item Find the maximal integral curve of $X=x\pdrv{}{x} + y \pdrv{}{y}$ starting at $(1,1) \in \R^2$.
\item Find the maximal integral curve of $Y=\pdrv{}{x} + y \pdrv{}{y}$ starting at $(1,1) \in \R^2$.
\end{enumerate}


\item \label{torus} $^{*}$$\R^{2}$に対し同値関係$\sim$を
$$
(x_1, y_1)\sim (x_2, y_2) \Leftrightarrow x_1 - x_2 \in \Z \text{かつ} y_1 - y_2 \in \Z 
$$
で定め, 2次元トーラス$T^2 := \R^2/\sim$とする. $\pi : \R^2 \rightarrow T^2$という商写像により$T^2$に位相を入れる. \footnote{$T^2 = \R^2/\Z^2$ともかく. この問題では$T^2$が$C^{\infty}$級多様体であることを認めて良い. また$T^2$は$S^{1} \times S^{1}$と微分同相である. } 
次の問いに答えよ.
\begin{enumerate}
\item $Y=\pdrv{}{x}$を$\R^2$上の$C^{\infty}$級ベクトル場とする. このとき$T^2$上の$C^{\infty}$級ベクトル場$X$で, 任意の$p \in \R^2$について$(d \pi)_{p} (Y)= X_{\pi(p)}$となるものが存在することを示せ.
\item $X$が生成する1パラメーター変換群$\{ \varphi_{t} \}_{t \in \R}$を求めよ.
\item  $T(T^{2})$は$ T^{2} \times \R$と微分同相であることを示せ. 
\end{enumerate}

%Let $X$ be the vector field $-y \pdrv{}{x} + x \pdrv{}{y}$ on $\R^{2}$.  Find the maximal integral curve of $X$ starting at $(1,0) \in \R^2$.
%\item Araujoの例




\item コンパクト$C^{\infty}$級多様体$M$上の任意の$C^{\infty}$級ベクトル場は完備であることを示せ.

\item  定理\ref{Lie_derivative}の(1)-(3)をそれぞれ示せ. 



\vspace{11pt}
\hspace{-33pt}{\large $\bullet$発展課題}
%\subsection{発展課題}

\hspace{-22pt}以下の問題は私が少々気になった事柄である. 余裕のある人向けの問題となっております. \footnote{教育的な問題からそうでない問題まで揃えております.}
%この授業とあまり関係ないあるいはこの授業の理解を大きく阻害をする問題であるため, あまり熱心にとかないことをお勧めする. 

\item $^{*}$ $C^{\infty}(M)$を$M$上の$C^{\infty}$級関数全体のなす集合とする. 
次の問いに答えよ.\footnote{必要であれば多様体の基礎 命題13.11を用いて良い.}
\begin{enumerate}
\item $C^{\infty}$級ベクトル場$X$について$D_{X} : C^{\infty}(M)  \rightarrow  C^{\infty}(M) $を$D_{X}(f) := Xf$で定める. $D_{X}$が線形かつライプニッツ則を満たすことを示せ. 
\item 写像$D :  C^{\infty}(M)  \rightarrow  C^{\infty}(M) $が線形でありライプニッツ則を満たすとき, ある$C^{\infty}$級ベクトル場$X$があって$D = D_{X}$となることを示せ.
\item $C^{\infty}$級ベクトル場$X,Y$について$X = Y$であることは$D_{X} = D_{Y}$であることと同値であることを示せ. \footnote{多様体の基礎 命題16.5の証明を見ていると, この本ではこの事実を認めている気がする. }
\end{enumerate}
ここで「線形」と「ライプニッツ則」については次のように定義する.
\begin{itemize}
\item $D$が線形であるとは$a,b \in \R, f,g \in C^{\infty}(M)$について$D(af + bg)=aD(f) + bD(g)$であることとする.
\item $D$がライプニッツ則を満たすとは$ f,g \in C^{\infty}(M)$について$D(fg)=D(f)g + fD(g)$であることとする.
\end{itemize}



%$X : C^{\infty}(M) \rightarrow C^{\infty}(M)$が線形でありライプニッツ則を満たすとき, $X$が$C^{\infty}$級ベクトル場であることを示せ. 

%ここで「$X$が線形であるとは$a,b \in \R, f,g \in C^{\infty}(M)$について$X(af + bg)=aX(f) + bX(g)$である」こととし, 
%「$X$がライプニッツ則を満たすとは$ f,g \in C^{\infty}(M)$について$X(fg)=X(f)g + fX(g)$である」こととする.

\item $f : M \rightarrow N$を$C^{\infty}$級写像とする. 
$C^{\infty}(M)$を$M$上の$C^{\infty}$級関数全体の集合として, 
$$
\begin{matrix}
f^{*} : &  C^{\infty}(N) &  \rightarrow & C^{\infty}(M) \\
	 &  \xi &  \mapsto & \xi \circ f\\
\end{matrix}
$$
と定める.  $X \in \mathscr{X}(M)$, $Y \in \mathscr{X}(N)$について$X$と$Y$が$f$-関係にあるとは
$D_{X} \circ f^{*} = f^{*} \circ D_Y$であることとする. 次の問いに答えよ.
\begin{enumerate}
\item $X$と$Y$が$f$-関係にあることは, 任意の$p \in M$について$(df)_{p}(X_{p}) = Y_{f(p)}$であることと同値であることを示せ.
\item $f$が微分同相写像のとき, 任意の$X \in \mathscr{X}(M)$について, $X$と$f$-関係にあるベクトル場$Y$がただ一つ存在することを示せ.
\item $X_1$と$Y_1$が$f$-関係にあり, $X_2$と$Y_2$が$f$-関係にあるとき, $[X_1, X_2]$と$[Y_1, Y_2]$も$f$-関係にあることをしめせ.
\end{enumerate}

\item  $M=N=\R$とし, $f : M \rightarrow N$を$f(x) = x^{\frac{1}{3}}$とする. 
$M$は$\R$への通常の$C^{\infty}$級多様体の構造を入れる. 
また$N$には$f^{-1} : N \rightarrow M = \R$によって$C^{\infty}$級多様体の構造を入れる
(つまり$\{ (N,f^{-1}) \}$が$N$の座標近傍系となる).
次の問いに答えよ.
\begin{enumerate}
\item $\varphi : M \rightarrow N$を恒等写像とする. $\varphi$は全単射な$C^{\infty}$級写像であることを示せ%$f^{-1} : N \rightarrow M$%$g : N \rightarrow M$を$g(y) = y^{3}$とおくと$g$
\item $\varphi^{-1}$は$C^{\infty}$級写像ではないことを示せ.つまり$\varphi$は$C^{\infty}$級微分同相ではない. 
\item $X = \pdrv{}{x}$について$\varphi$-関係にある$N$上の$C^{\infty}$級ベクトル場は存在しないこと示せ. 
\end{enumerate}

%\item 定理\ref{Lie_derivative}の(1)-(4)を証明せよ.

\item $^{**}$ $TS^3 $は$S^3 \times \R^{3}$と微分同相であることを示せ. \footnote{ヒント: 四元数体のノルム1の全体集合が$S^{3}$になる.}
\item$^{***} $ $TS^n$が$S^n \times \R^{n}$と微分同相となるような自然数$n$を決定せよ. \footnote{$n$が偶数ではないことはPoincare-Hopfの定理からわかる. $n$が奇数のときどのように議論するか私はわからない.}

%\item 定理\ref{tangent_vector_space}の事柄を証明せよ. 特に証明のどこに$C^{\infty}$を使ったのかを明らかにせよ. \footnote{この問題は解く意味があまりないかもしれない.}

\end{enumerate}


\end{comment}
%%%%%%%%%%%%%%%%%%%%%%%%%%%%%%


