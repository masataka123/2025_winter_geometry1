\documentclass[dvipdfmx,a4paper,11pt]{article}
\usepackage[utf8]{inputenc}
%\usepackage[dvipdfmx]{hyperref} %リンクを有効にする
\usepackage{url} %同上
\usepackage{amsmath,amssymb} %もちろん
\usepackage{amsfonts,amsthm,mathtools} %もちろん
\usepackage{braket,physics} %あると便利なやつ
\usepackage{bm} %ラプラシアンで使った
\usepackage[top=15truemm,bottom=30truemm,left=25truemm,right=25truemm]{geometry} %余白設定
\usepackage{latexsym} %ごくたまに必要になる
\renewcommand{\kanjifamilydefault}{\gtdefault}
\usepackage{otf} %宗教上の理由でmin10が嫌いなので
%\usepackage{showkeys}\renewcommand*{\showkeyslabelformat}[1]{\fbox{\parbox{2cm}{ \normalfont\tiny\sffamily#1\vspace{6mm}}}}
%\usepackage[driverfallback=dvipdfm]{hyperref}


\usepackage[all]{xy}
\usepackage{amsthm,amsmath,amssymb,comment}
\usepackage{amsmath}    % \UTF{00E6}\UTF{0095}°\UTF{00E5}\UTF{00AD}\UTF{00A6}\UTF{00E7}\UTF{0094}¨
\usepackage{amssymb}  
\usepackage{color}
\usepackage{amscd}
\usepackage{amsthm}  
\usepackage{wrapfig}
\usepackage{comment}	
\usepackage{graphicx}
\usepackage{setspace}
\usepackage{pxrubrica}
\usepackage{enumitem}
\usepackage{mathrsfs} 

\setstretch{1.2}


\newcommand{\R}{\mathbb{R}}
\newcommand{\Z}{\mathbb{Z}}
\newcommand{\Q}{\mathbb{Q}} 
\newcommand{\N}{\mathbb{N}}
\newcommand{\C}{\mathbb{C}} 
\newcommand{\Sin}{\text{Sin}^{-1}} 
\newcommand{\Cos}{\text{Cos}^{-1}} 
\newcommand{\Tan}{\text{Tan}^{-1}} 
\newcommand{\invsin}{\text{Sin}^{-1}} 
\newcommand{\invcos}{\text{Cos}^{-1}} 
\newcommand{\invtan}{\text{Tan}^{-1}} 
\newcommand{\Area}{\text{Area}}
\newcommand{\vol}{\text{Vol}}
\newcommand{\maru}[1]{\raise0.2ex\hbox{\textcircled{\tiny{#1}}}}
\newcommand{\sgn}{{\rm sgn}}
%\newcommand{\rank}{{\rm rank}}



   %当然のようにやる.
\allowdisplaybreaks[4]
   %もちろん.
%\title{第1回. 多変数の連続写像 (岩井雅崇, 2020/10/06)}
%\author{岩井雅崇}
%\date{2020/10/06}
%ここまで今回の記事関係ない
\usepackage{tcolorbox}
\tcbuselibrary{breakable, skins, theorems}

\theoremstyle{definition}
\newtheorem{thm}{定理}
\newtheorem{lem}[thm]{補題}
\newtheorem{prop}[thm]{命題}
\newtheorem{cor}[thm]{系}
\newtheorem{claim}[thm]{主張}
\newtheorem{dfn}[thm]{定義}
\newtheorem{rem}[thm]{注意}
\newtheorem{exa}[thm]{例}
\newtheorem{conj}[thm]{予想}
\newtheorem{prob}[thm]{問題}
\newtheorem{rema}[thm]{補足}

\DeclareMathOperator{\Ric}{Ric}
\DeclareMathOperator{\Vol}{Vol}
 \newcommand{\pdrv}[2]{\frac{\partial #1}{\partial #2}}
 \newcommand{\drv}[2]{\frac{d #1}{d#2}}
  \newcommand{\ppdrv}[3]{\frac{\partial #1}{\partial #2 \partial #3}}


%ここから本文.
\begin{document}
\pagestyle{empty}

 
  \begin{center}
 {\Large 中間レポート2 提出用紙 }
% {\Large 演習問題の解答用紙 2024年1月11日(木) } \\
%\end{center}
%\vspace{5pt}

{ \large 提出締め切り 2025年1月24日(金) 15時10分00秒 (日本標準時刻)}
\end{center}

%\vspace{2pt}
\begin{flushleft}
{ \large \underline{学籍番号: \hspace{4cm} 名前  \hspace{9cm}   }  }
\end{flushleft}

\begin{flushleft}
{ \Large 提出方法}
\end{flushleft}
\begin{itemize}
 \setlength{\parskip}{0cm}
  \setlength{\itemsep}{2pt} 
  \item 1月24日(金) の授業を始める時に「紙で提出する」もしくは「電子媒体(PDF)などでCLEから提出」してください. 
  紙で提出する場合はこの用紙を表紙にしてホッチキスで左上をとめて提出すること.
\item 解答に関しては答えのみならず, 答えを導出する過程をきちんと記すこと. 
\item レポート問題に関してはCLEに解答があるのでそれを活用してよい. ただし意味もなく丸写ししても時間の無駄なので, 使う際はなぜその解答になるのか考えながら活用すること. 
\end{itemize}

\begin{flushleft}
{\Large レポート問題}
\end{flushleft}
\begin{enumerate}[label=\textbf{問題}\arabic*]

\item (演習問題 4.1) 
\label{circle}  $S^1 \subset \R^2$を円周とする. 
$$
U = S^1 \setminus \{(1,0)\} \quad \varphi_U(\cos\theta_U , \sin \theta_U) =\theta_U \quad (0 < \theta_U < 2\pi)
$$
$$
V= S^1 \setminus \{(-1,0)\} \quad \varphi_V(\cos\theta_V, \sin \theta_V) =\theta_V \quad (-\pi < \theta_V < \pi)
$$
として座標近傍$(U, \varphi_U), (V, \varphi_V)$を定める.
$U$上の1次微分形式$\alpha_U$と$V$上の1次微分形式$\alpha_V$を
$$
\alpha_U = d \theta_U, \quad \alpha_U = d \theta_V 
$$
とする. %\footnote{厳密には$d \theta_U$は$\varphi(U)$上の微分形式と同一視している.} 
このとき$U \cap V$上で$\alpha_U = \alpha_V$であることを示せ. 
これにより$S^1$上の微分形式$\alpha$を
$$
\alpha_x = \left\{
\begin{array}{ll}
(\alpha_U)_x  & (x \in  U)\\
(\alpha_V)_x & (x \in V)
\end{array}
\right.
$$
として定めることができる. 
	
\item (演習問題 4.2) 
$\varphi: S^1 \to \R^2$を包含写像とし, $\alpha$を\ref{circle}での$S^1$上の1次微分形式とする. このとき以下が成り立つことを示せ. 
$$
\varphi^{*}\left( \frac{-ydx + x dy}{x^2+y^2} \right) = \alpha
$$

\item (演習問題 4.4) 
$i : S^2 \rightarrow \R^3$を包含写像とする. 次の問いに答えよ.
\begin{enumerate}
 \setlength{\parskip}{0cm}
  \setlength{\itemsep}{2pt} 
\item $i^{*}(dx \wedge dy \wedge dz)$を求めよ.
\item $B = \{ (u,v) \in \R^2 | u^2 + v^2 < 1\}$とし, $\varphi : B \to S^2$を$\varphi(u,v)= (u, \sqrt{1 - u^2 - v^2}, v)$とする. $(i \circ \varphi)^{*}(dx \wedge dy)$の値が0になる$B$の点を全て求めよ. 
\item $i^{*}(dx \wedge dy)$の値が0になる$S^2$の点を全て求めよ.
\end{enumerate}

\item (演習問題 5.2) 
\ref{circle}において定義した$S^1$上の1次微分形式$\alpha$について, $\int_{S^1 } \alpha$の値を求めよ.


\item (演習問題 5.4) 
$\int_{S^2} x dy \wedge dz + y dz \wedge dx + z dx \wedge dy$を求めよ. 


 \end{enumerate}

 \end{document}
